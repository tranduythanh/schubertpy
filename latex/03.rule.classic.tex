\section{Các quy tắc tính toán trong phép tính Schubert cổ điển}

\subsection{Quy tắc Pieri}
Một trường hợp đặc biệt quan trọng của bài toán nhân các lớp Schubert là nhân một lớp Schubert bất kỳ với một lớp đặc biệt biểu diễn bằng một phân hoạch có dạng một hàng hoặc một cột. Những lớp tương ứng thường được gọi là \textit{các lớp đặc biệt} (special Schubert classes), ví dụ $\sigma_{(a)}$ (một hàng dài $a$ ô) hoặc $\sigma_{(1^a)}$ (một cột cao $a$ ô) trong $H^*(Gr(k,n))$. \textit{Công thức Pieri} cung cấp cách tính đơn giản cho trường hợp này. Phát biểu đơn giản của quy tắc Pieri:
\[ \sigma_{\lambda} \cdot \sigma_{(a)} \;=\; \sum_{\nu} \sigma_{\nu}, \]
trong đó tổng lấy qua tất cả các phân hoạch $\nu$ thu được bằng cách thêm $a$ ô không nằm cùng một cột vào biểu đồ (diagram) của $\lambda$. Tương tự,
\[ \sigma_{\lambda} \cdot \sigma_{(1^a)} \;=\; \sum_{\nu} \sigma_{\nu}, \]
trong đó $\nu$ thu được bằng cách thêm $a$ ô không nằm cùng một hàng vào biểu đồ của $\lambda$. Những phân hoạch $\nu$ thu được theo cách trên đảm bảo $\nu$ vẫn phù hợp với $Gr(k,n)$ (không vượt quá kích thước lưới $k \times (n-k)$ của biểu đồ Young). Công thức Pieri thực chất cho ta một quy tắc tổ hợp để liệt kê các lớp xuất hiện khi nhân với một lớp đặc biệt, và tất cả các hệ số trong hai biểu thức trên đều bằng 1. Mặc dù đơn giản, công thức Pieri đủ mạnh để giải quyết nhiều bài toán đếm, đặc biệt là những bài toán trong đó một trong các điều kiện tương ứng với một phân hoạch dạng hàng hoặc cột (ví dụ bài toán "có bao nhiêu đường thẳng đi qua một điểm cố định và thỏa các điều kiện khác..." v.v.).

\subsection{Công thức Giambelli}
Pieri giải quyết trường hợp nhân với lớp đặc biệt. Để biểu diễn một lớp Schubert bất kỳ thành phần tử của vành đa thức sinh bởi các lớp đặc biệt, \textit{công thức Giambelli} cung cấp mối liên hệ cần thiết. Công thức Giambelli phát biểu rằng mỗi lớp Schubert $\sigma_{\lambda}$ trong $H^*(Gr(k,n))$ có thể biểu diễn như một định thức (determinant) với các phần tử là các lớp đặc biệt. Cụ thể, nếu 
$$\lambda = (\lambda_1,\lambda_2,\dots,\lambda_r)$$ 
có $r$ phần, khi đó:
$$ \sigma_{\lambda} = \det\big[\sigma_{(\lambda_i + j - i)}\big]_{1 \le i,j \le r}.
$$
Ở đây vế phải là định thức của ma trận $r \times r$ với phần tử hàng $i$, cột $j$ là lớp đặc biệt $\sigma_{(\lambda_i + j - i)}$. Ví dụ, 
$$\sigma_{(a,b)} = \sigma_{(a)}\cdot \sigma_{(b-1)} - \sigma_{(a-1)}\cdot \sigma_{(b)}$$ 
với $a \ge b$ là một trường hợp cụ thể của công thức trên. Công thức Giambelli cho phép ta tạo một cầu nối giữa các lớp Schubert tổng quát và các lớp đặc biệt, từ đó mọi phép tính trên các lớp Schubert có thể quy về tính toán với các lớp đặc biệt (mà ta đã có quy tắc Pieri để xử lý).

\subsection{Quy tắc Littlewood-Richardson}
Trường hợp tổng quát nhất---nhân hai lớp Schubert bất kỳ với nhau $\sigma_{\lambda} \cdot \sigma_{\mu}$---được giải quyết bằng \textit{quy tắc Littlewood-Richardson (L--R)}. Quy tắc L--R là một kết quả cơ bản trong đại số tổ hợp và lý thuyết biểu diễn, cung cấp một cách đếm thuần túy tổ hợp để xác định các hệ số $c_{\lambda,\mu}^{\,\nu}$ trong tích Schubert. Phát biểu của quy tắc L--R sử dụng ngôn ngữ của \textit{biểu đồ Young} và \textit{bảng Young} (Young tableau). Ý tưởng chính như sau: một \textit{bảng Young nửa chuẩn tắc (semi-standard Young tableau, SSYT)} của dạng hình (shape) $\nu/\lambda$ (biểu đồ của $\nu$ bỏ đi biểu đồ của $\lambda$) với \textit{nội dung} (content) $\mu$ được định nghĩa là cách điền các số tự nhiên vào các ô của hình $\nu/\lambda$ sao cho các hàng không giảm từ trái sang phải và các cột tăng từ trên xuống dưới, và số lượng các ô chứa số $i$ đúng bằng $\mu_i$ với $\mu = (\mu_1,\mu_2,\dots)$. Một bảng như vậy được gọi là \textit{bảng Littlewood-Richardson} nếu thỏa thêm điều kiện đặc biệt gọi là \textit{điều kiện Littlewood-Richardson}: khi đọc các ô theo thứ tự từ phải sang trái, từ dòng trên xuống dòng dưới (được gọi là \textit{từ LR}), tại mọi điểm của quá trình đọc, số lần xuất hiện của $j$ không bao giờ vượt quá số lần xuất hiện của $j+1$ đối với mọi $j$. Số lượng các bảng Littlewood-Richardson dạng $\nu/\lambda$ với nội dung $\mu$ chính là hệ số $c_{\lambda,\mu}^{\,\nu}$.

Quy tắc Littlewood-Richardson tuy trừu tượng hơn, nhưng cung cấp một công cụ cực kỳ mạnh để tính toán trong vành đồng điều của $Gr(k,n)$ và giải các bài toán hình học liệt kê. Từ quy tắc này, ta có thể suy ra cả công thức Pieri (là trường hợp đặc biệt khi $\mu$ là một hàng hoặc một cột duy nhất) lẫn công thức Giambelli (khi kết hợp với tính đối ngẫu Poincaré trong vành đồng điều). Trong thực tế, việc áp dụng quy tắc L-R có thể khá phức tạp vì số lượng bảng Young cần xét có thể rất nhiều khi kích thước các phân hoạch tăng. Tuy nhiên, nó đảm bảo rằng về mặt lý thuyết, mọi phép nhân Schubert đều có thể giải được một cách triệt để.

Tóm lại, ba kết quả kinh điển---Pieri, Giambelli và Littlewood-Richardson---tạo thành nền tảng tính toán cho phép tính Schubert cổ điển. Chúng cho phép chuyển mọi bài toán đếm hình học tuyến tính cổ điển (trên các Grassmannian) về một bài toán tổ hợp về biểu đồ Young. Những công cụ này được phát triển trong nửa đầu thế kỷ 20 (Pieri 1890, Giambelli 1902, Littlewood-Richardson 1934) đã giúp hoàn thành phần lớn chương trình đặt cơ sở chặt chẽ cho phép tính Schubert như Hilbert đề ra, kết hợp với sự phát triển của lý thuyết đồng điều và hình học đại số hiện đại (các kết quả của Ehresmann, Chow, Chevalley và các nhà toán học khác).
