
\section{Các hướng mở rộng hiện đại}
\subsection{Đồng điều lượng tử hiệp biến}
Phép tính Schubert lượng tử, sau hơn ba thập kỷ phát triển, vẫn là một lĩnh vực sôi động với nhiều hướng nghiên cứu mở rộng. Một trong những hướng đó là việc tổng quát hóa phép tính Schubert lượng tử sang \textit{trường hợp hiệp biến} (equivariant quantum cohomology), nơi người ta xét đồng thời tác động của một nhóm đối xứng lên không gian. Đồng điều lượng tử hiệp biến đòi hỏi kết hợp cả lý thuyết đặc trưng Chern và tích phân trên moduli, dẫn đến các kết quả phức tạp hơn nhưng cũng giàu thông tin hơn về không gian (ví dụ: công trình tính toán $QH_T^*(G/P)$ cho các đa tạp cờ $G/P$ dưới tác động của torus). 

\subsection{Lý thuyết K}
Hướng thứ hai là mở rộng sang \textit{lý thuyết K lượng tử} và xa hơn là \textit{đồng điều elliptic lượng tử}, tức là xây dựng các phiên bản của phép tính Schubert trong lý thuyết K (đối tượng là các lớp bó vector (vector bundle)) hoặc trong lý thuyết elliptic (liên quan đến các hàm theta và dạng module). Những mở rộng này gắn liền với việc đếm không chỉ các đối tượng hình học đơn giản mà cả các bó vector holomorphic trên đường cong, và có liên hệ với các giả thuyết vật lý mới (ví dụ: đối ngẫu lượng tử – K và đối xứng gương ở mức độ K). Về mặt tổ hợp, các nhà nghiên cứu tiếp tục tìm kiếm và chứng minh các \textit{tính chất tương tự cổ điển} cho các hệ số Gromov–Witten: chẳng hạn, tính \textit{dương} và \textit{nguyên} của các hệ số trong phép nhân lượng tử (tương tự tính dương của hằng số Littlewood–Richardson cổ điển) đã được chứng minh cho nhiều trường hợp, cho thấy ý nghĩa đại số sâu sắc của những con số đếm đường cong này \cite{XXX}. 

\subsection{Các hướng khác}
Ngoài ra, phép tính Schubert lượng tử còn mở rộng sang các không gian mới ngoài phạm vi cờ cổ điển: ví dụ, tính toán đồng điều lượng tử cho các \textit{đa tạp Grassmann đẳng hướng} (loại B, C, D) và các \textit{đa tạp cờ tổng quát} (bao gồm cả các trường hợp nhóm Lie ngoại lệ) đang được tiến hành, đòi hỏi kết hợp nhiều kỹ thuật từ đại số tổ hợp đến hình học đại số hiện đại. Một hướng liên ngành khác là mối quan hệ giữa \textit{tính khả tích} (integrability) và đồng điều lượng tử: phương trình \textit{vi phân lượng tử} (xuất phát từ việc xét hàm thế Gromov–Witten) thường trùng với các hệ khả tích cổ điển (như hệ Toda), gợi mở mối liên hệ giữa phép tính Schubert lượng tử và lý thuyết soliton, hàm tau,...

