\section{Phép tính Schubert lượng tử}
\subsection{Mở rộng phép tính Schubert}
Đến thập niên 1990, dưới sự ảnh hưởng của vật lý lý thuyết (đặc biệt là lý thuyết dây), khái niệm \textit{phép tính Schubert lượng tử} ra đời như một mở rộng hiện đại của phép tính Schubert. Phép tính Schubert lượng tử đưa thêm vào bức tranh cổ điển thông tin về các \textit{đường cong đại số} (đặc biệt là các đường cong ``hữu tỉ'' – tương đương hình học với $\mathbb{CP}^1$) bên trong không gian. Thông qua lý thuyết \textit{đồng điều lượng tử}, người ta định nghĩa một phép nhân mới (gọi là \textit{phép nhân lượng tử}) trên các lớp Schubert sao cho ngoài các giao cắt thông thường, còn tính đến khả năng các lớp này \textit{giao nhau} khi được \textit{nối với nhau bởi một hay nhiều đường cong}. Nói cách khác, nếu trong đồng điều cổ điển hai lớp Schubert $\sigma_u$ và $\sigma_v$ chỉ giao phiếm hàm khi chúng thực sự cắt nhau, thì trong đồng điều lượng tử, $\sigma_u$ và $\sigma_v$ còn có thể cho kết quả khác $0$ nếu tồn tại đường cong (ví dụ: một đường cong hữu tỉ) nằm trong không gian xét, nối liền các vị trí mà hai lớp này ánh xạ tới. Hệ số cho những \textit{giao điểm lượng tử} này chính là các \textit{bất biến Gromov–Witten}---về cơ bản là số lượng (đếm có hướng hoặc đếm xấp xỉ) các đường cong thỏa mãn những điều kiện giao hình học tương ứng. Kết quả, phép tính Schubert lượng tử thiết lập một cấu trúc đại số mới trên không gian đồng điều: \textit{vành đồng điều lượng tử (nhỏ)} $QH^*(X)$, trong đó $X$ là không gian xét (ví dụ $X$ có thể là đa tạp cờ hoặc Grassmann). Vành này mở rộng vành đồng điều thông thường bằng cách thêm các biến hình thức (thường ký hiệu $q$) mang thông tin về độ của đường cong, và phép nhân chén thông thường được \textit{biến dạng} (deform) thành phép nhân lượng tử có kèm các hệ số $q$.

Vành đồng điều lượng tử của Grassmannian $Gr(k,n)$ có dạng:
$$
QH^\star(Gr(k,n)) = \bigoplus_{\lambda \subseteq k \times (n-k)} \mathbb{Z}[q] \cdot \sigma_\lambda,
$$
với $q$ là biến lượng tử có bậc $n$ (tương ứng với số điều kiện cần để có đường cong bậc 1 trong không gian đó). Trong $QH^*(X)$ người ta định nghĩa tích lượng tử:
$$
\sigma_u * \sigma_v = \sum_{d \geq 0} \sum_w N^{w,d}_{u,v} \, q^d \, \sigma_w,
$$
trong đó
\begin{itemize}
    \item $\sigma_u, \sigma_v, \sigma_w$ là các lớp Schubert (hay nói chung là các lớp đồng điều) trong không gian $X$. 
    \item Hệ số $N^{w,d}_{u,v}$ là các bất biến Gromov–Witten 3 điểm (genus 0) của $X$, trực tiếp \textit{đếm số lượng đường cong hữu tỉ cấp $d$} trong $X$ đồng thời thỏa mãn ba điều kiện: đi qua vị trí tổng quát thuộc các lớp Schubert $\sigma_u, \sigma_v$ và $\sigma_w$ (chỉ định thông qua tính chất giao với cờ cố định). 
\end{itemize}
Khi $d=0$ (đường cong cấp 0 có thể hiểu là giao điểm thông thường), $N^{w,0}_{u,v}$ chính là số giao điểm hữu hạn thông thường – trùng với hằng số cấu trúc cổ điển. Bởi vậy, nếu bỏ qua các bội $q^d$ với $d>0$, công thức trên thu về phép nhân Schubert cổ điển. Ngược lại, với $d>0$, $N^{w,d}_{u,v}$ cho ta thông tin về \textit{bài toán đếm đường cong}: ví dụ, trong không gian cờ cỡ nhỏ, một số bất biến Gromov–Witten quen thuộc đếm số đường conic hoặc đường cong bậc cao hơn thỏa mãn các điều kiện hình học nhất định. Các \textit{đa thức Schubert lượng tử} được xây dựng để biểu diễn các lớp Schubert trong vành $QH^*(X)$ cũng như tính toán hiệu quả phép nhân lượng tử. Chẳng hạn, trong trường hợp $X$ là đa tạp Grassmann, các đa thức Schubert lượng tử (do Fomin, Gelfand và Postnikov đề xuất) mở rộng đa thức Schubert cổ điển bằng cách thêm các biến $q$ và thỏa mãn các quan hệ mới phản ánh các đếm đường cong (ví dụ: các công thức \textit{Pieri lượng tử} và \textit{Giambelli lượng tử}). Điều này cho phép ta tính toán nhanh chóng phép nhân lượng tử $\sigma_u * \sigma_v$ bằng cách nhân các đa thức tương ứng rồi ánh xạ kết quả theo các quan hệ lượng tử (ví dụ: với Grassmann $Gr(k,n)$, quan hệ cơ bản thường có dạng $h^n = q$ với $h$ là lớp hyperplane thích hợp) \cite{buch2001quantumcohomologygrassmannians}. 

\subsection{Vai trò trong toán học}
Phép tính Schubert lượng tử không chỉ nối tiếp truyền thống \textit{hình học liệt kê} của Schubert mà còn mở ra các hướng nghiên cứu mới trong toán học hiện đại. Trước hết, nó cung cấp một công cụ mạnh để giải quyết các bài toán đếm hình học phức tạp từng nằm ngoài tầm với của phương pháp cổ điển. Ví dụ, bằng cách sử dụng đồng điều lượng tử, các nhà toán học đã tính được số đường cong bậc $d$ đi qua các vị trí cho trước trên nhiều đa tạp (một bài toán có liên hệ tới giả thuyết đường cong cong của vật lý). Kết quả tiên đoán bởi vật lý về số lượng đường cong trên các đa tạp đặc biệt (như đa tạp Calabi–Yau) đã được xác nhận thông qua tính toán Gromov–Witten, cho thấy sức mạnh dự đoán của phép tính Schubert lượng tử. Hơn nữa, cấu trúc đại số của $QH^*(X)$ (một \textit{vành giao hoán có đơn vị} với phép nhân biến dạng) mang tính chất của một \textit{đại số Frobenius}, gắn liền với hiện tượng đối ngẫu gương và các hệ phương trình vi phân khả tích (phương trình $q$-hệ số WDVV) trong hình học. Trong một số trường hợp đặc biệt, đồng điều lượng tử của đa tạp đồng nhất trùng với các cấu trúc nổi tiếng khác trong toán học: chẳng hạn Edward Witten đã chỉ ra rằng $QH^*(Gr(k,n))$ tương đương với \textit{đại số Verlinde} mô tả phép tính fusion trong lý thuyết nhóm con hữu hạn \cite{witten1993verlindealgebracohomologygrassmannian}. Điều này giải quyết một cách hình học bài toán đếm của lý thuyết biểu diễn (tính kích thước không gian các hàm suy rộng) và cũng đồng thời giải thích một hiện tượng vật lý 2 chiều (tương ứng với mô hình trường $WZW$ trên mặt cầu) bằng ngôn ngữ hình học đại số. Một ứng dụng nổi bật khác là lời giải cho \textit{bài toán phổ điểm riêng} trong đại số tuyến tính: thông qua phép tính Schubert lượng tử trên $Gr(k,n)$, Agnihotri và Woodward đã mô tả đầy đủ các bất đẳng thức đặc trưng cho khả năng tồn tại ma trận đơn vị với phổ cho trước (liên quan đến \textit{dự đoán Horn} về cộng các phổ ma trận) \cite{agnihotri1997eigenvaluesproductsunitarymatrices}. Kết quả này kết nối bất ngờ giữa hình học (các bất biến Gromov–Witten trên cờ) và đại số tuyến tính, đồng thời khẳng định tính \textit{dương} của mọi bất biến Gromov–Witten (một tính chất quan trọng được dự đoán trước đó). Như vậy, phép tính Schubert lượng tử đã liên kết các lĩnh vực tưởng chừng rất xa nhau – từ hình học đại số, tổ hợp cho đến lý thuyết biểu diễn và đại số tuyến tính – tạo nên một bức tranh đa ngành phong phú.

\subsection{Vai trò trong vật lý lý thuyết}
Phép tính Schubert lượng tử (và lý thuyết đồng điều lượng tử nói chung) có nguồn gốc sâu xa từ vật lý lý thuyết, cụ thể là từ \textit{lý thuyết trường lượng tử topo} và \textit{lý thuyết dây}. Các bất biến Gromov–Witten ban đầu xuất hiện như các \textit{biên độ xác suất} trong lý thuyết topological sigma-model A (mô tả các mặt cầu holomorphic ánh xạ vào một không gian đích $X$). Việc \textit{đếm} các đường cong chỉnh hình (holomorphic) này trong vật lý tương đương với việc tính các tích phân trên \textit{không gian moduli} của các ánh xạ ổn định---đó chính là cách các nhà toán học định nghĩa nghiêm túc bất biến Gromov–Witten. Do đó, phép tính Schubert lượng tử là cầu nối giữa toán học và vật lý: mỗi đẳng thức trong đồng điều lượng tử (chẳng hạn $h^{n} = q$ trong $QH^*(\mathbb{CP}^{n-1})$) có thể diễn giải như một quá trình vật lý (ở đây là $n$ hạt điểm giao nhau có thể được thay thế bằng một hạt điểm duy nhất nếu có sự xuất hiện của một \textit{instanton} là một cầu phương vị---chính là đường cong $\mathbb{CP}^1$---đóng vai trò kết nối). Đặc biệt, \textit{đối xứng gương}---mối liên hệ huyền bí giữa hai không gian hình học tưởng chừng khác biệt – được kiểm chứng và nghiên cứu thông qua đồng điều lượng tử: vành đồng điều lượng tử của $X$ (A-model) tương ứng với cấu trúc giải tích phức trên \textit{gương} của $X$ (B-model), và ngược lại \cite{nlab:gromov-witten_invariants}. Nhiều bài toán đếm đường cong khó (ví dụ: đếm số đường cong trên một đa tạp Calabi–Yau 3 chiều như quintic) đã được giải quyết nhờ đối xứng gương, từ đó thúc đẩy sự phát triển của phép tính Schubert lượng tử và lý thuyết Gromov–Witten trong toán học. Ngày nay, các bất biến Gromov–Witten còn được xem là một phần của các \textit{đại số lượng tử} trong lý thuyết trường, liên hệ mật thiết với các đại lượng vật lý như \textit{hàm phân hoạch} (partition function) và \textit{hàm tương quan} của lý thuyết dây. Sự tương đồng cấu trúc giữa vành đồng điều lượng tử và những lý thuyết vật lý 2 chiều cho phép hai ngành học hỏi lẫn nhau: vật lý gợi ý nhiều giả thuyết hình học táo bạo (như đối xứng gương, giả thuyết về tính toàn vẹn Gromov–Witten), trong khi các kết quả toán học cung cấp cơ sở chặt chẽ để hiểu rõ hơn các mô hình vật lý.
