\section{Giới thiệu}
Hình học luôn gắn liền với việc nghiên cứu các đối tượng hình dạng, kích thước, vị trí tương đối và các tính chất của không gian. Một khía cạnh cơ bản của nghiên cứu này là khả năng \textit{đếm} các đối tượng hình học thỏa mãn những điều kiện cho trước. Chẳng hạn, một câu hỏi kinh điển: 

\epigraph{\textit{``Có bao nhiêu đường thẳng trong không gian xạ ảnh ba chiều $\mathbb{P}^3$ cắt bốn đường thẳng cho trước ở vị trí tổng quát?''}} 

Câu hỏi tưởng chừng đơn giản này thực ra có đáp số không tầm thường: chính xác là \textit{hai} đường thẳng như vậy . Những bài toán tương tự xuất hiện tự nhiên trong nhiều ngành toán học khác nhau và được gọi chung là các bài toán của \textit{hình học liệt kê} (enumerative geometry)---ngành nghiên cứu về đếm số lượng các đối tượng hình học thỏa mãn các điều kiện cho trước. Các vấn đề đếm này không chỉ có ý nghĩa thuần túy về toán học mà còn có liên hệ sâu sắc đến vật lý (ví dụ như lý thuyết dây) và khoa học máy tính.

Vào cuối thế kỷ 19, nhà toán học Hermann Schubert đã tiên phong xây dựng một tập hợp phương pháp để giải quyết nhiều bài toán hình học liệt kê phức tạp, đặc biệt là những bài toán liên quan đến các không gian con tuyến tính trong không gian xạ ảnh. Công trình của ông, tập hợp trong cuốn sách kinh điển \textit{Kalkül der abzählenden Geometrie} xuất bản năm 1879, đã đặt nền móng cho lĩnh vực ngày nay gọi là \textit{phép tính Schubert} \cite{schubert1879kalkul}. Phương pháp của Schubert tỏ ra cực kỳ hiệu quả trong việc đưa ra đáp số đúng cho nhiều bài toán hóc búa. Ông đã giới thiệu các ký hiệu hình học để biểu diễn các điều kiện vị trí và xây dựng những quy tắc biến đổi các ký hiệu đó nhằm thu được kết quả đếm số lượng. Tuy nhiên, theo tiêu chuẩn chặt chẽ của toán học vào đầu thế kỷ 20, các lập luận của Schubert chưa có nền tảng vững chắc về mặt lý thuyết. Điều này đã dẫn tới việc David Hilbert đưa ra bài toán thứ 15 trong danh sách 23 bài toán nổi tiếng của ông, yêu cầu xây dựng cơ sở toán học nghiêm ngặt cho phép tính liệt kê của Schubert \cite{Encyclo_schubert_calculus}.

Trong suốt thế kỷ 20, các nhà toán học đã nỗ lực giải quyết thách thức của Hilbert. Những công cụ hiện đại từ tô-pô đại số và lý thuyết giao (intersection theory) đã được phát triển để đặt phương pháp của Schubert trên nền tảng vững chắc \cite{Kontsevich_1994}. Kết quả là phép tính Schubert đã được xây dựng lại một cách chặt chẽ thông qua việc hiểu các phép đếm như những tích trong \textit{vành đồng điều} (cohomology) của các đa tạp cờ (ví dụ như các đa tạp Grassmann) \cite{BERTRAM1999728}. Ngày nay, phép tính Schubert không chỉ dừng lại ở các kết quả cổ điển của thế kỷ 19 mà còn mở rộng đáng kể phạm vi. Những phát triển hiện đại đã đưa đến \textit{phép tính Schubert lượng tử} (quantum Schubert calculus), phép tính Schubert trên $K$-theory, và các phiên bản equivariant (đối xứng hóa) \cite{ALCO_2018__1_3_327_0}. Những mở rộng này cho phép giải quyết các bài toán trước đây không thể tiếp cận được, đồng thời tiết lộ các kết nối sâu sắc giữa hình học liệt kê với những lĩnh vực khác như đại số tổ hợp, lý thuyết biểu diễn và vật lý lý thuyết \cite{Encyclo_schubert_calculus}.

Mặc dù đã đạt được những tiến bộ lý thuyết to lớn, việc tính toán trong phép tính Schubert vẫn là một thách thức không nhỏ. Với các không gian có độ phức tạp cao, việc tự tính toán thủ công các phép nhân lớp Schubert hay các hệ số Littlewood-Richardson dễ dẫn đến sai sót và mất nhiều thời gian \cite{graysonschubert2}. Do đó, nhu cầu về các \textit{công cụ tính toán} chuyên dụng cho phép tính Schubert là rất lớn. Hiện nay, một số phần mềm toán học như SageMath đã cung cấp một số chức năng liên quan đến đa thức Schubert \cite{sagemath}, tuy nhiên chúng chưa được thiết kế chuyên sâu cho các nhà nghiên cứu trong lĩnh vực này. Để đáp ứng nhu cầu đó, chúng tôi phát triển \textit{SchubertPy}, một gói phần mềm Python cung cấp các công cụ toàn diện để tính toán và thao tác với các đa thức và lớp Schubert. SchubertPy được lấy cảm hứng từ gói Maple có tên \textit{qcalc} của Anders Skovsted Buch \cite{buch2008qcalc}, nhưng được xây dựng trên nền tảng Python hiện đại nhằm tận dụng sự linh hoạt và sức mạnh của hệ sinh thái Python (bao gồm SymPy, SageMath, NumPy, v.v.).

Trong các phần tiếp theo của bài viết, chúng tôi sẽ lần lượt giới thiệu cơ sở lý thuyết của phép tính Schubert (bao gồm các khái niệm về đa tạp Grassmann, lớp Schubert và vành đồng điều), trình bày các quy tắc tính toán chính (như quy tắc Pieri, công thức Giambelli, quy tắc Littlewood–Richardson) kèm theo các ví dụ minh họa, và sau đó giới thiệu cách sử dụng công cụ SchubertPy để giải quyết các bài toán mẫu. Cuối cùng, chúng tôi thảo luận kết luận và đề xuất một số hướng phát triển mở rộng cho cả lý thuyết và công cụ này.