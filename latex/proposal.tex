\documentclass[conference,a4paper]{IEEEtran}
\usepackage[utf8]{inputenc}
\usepackage[vietnamese]{babel}
\usepackage{url}
\usepackage{xcolor}
\usepackage{hyperref}
\usepackage{fancyhdr}

\usepackage{epigraph}
\usepackage{amsfonts}
\usepackage{amsmath}
\usepackage{algorithm}
\usepackage{algorithmic}

\setlength\epigraphwidth{8cm}
\setlength\epigraphrule{0pt}

\pagestyle{fancy}

\fancyhf{}
\cfoot{\thepage}
\renewcommand{\headrulewidth}{0pt}

% Customize the color of citations
\hypersetup{
    colorlinks=true,
    citecolor=blue,  % Change this color as needed
    linkcolor=black,
    urlcolor=blue
}

\ifCLASSINFOpdf
\else
\fi

% correct bad hyphenation here
\hyphenation{op-tical net-works semi-conduc-tor}

\DeclareRobustCommand*{\IEEEauthorrefmark}[1]{\raisebox{0pt}[0pt][0pt]{\textsuperscript{\footnotesize #1}}}

\addto\captionsvietnamese{
  \renewcommand{\abstractname}{Abstract}
}

\addto\captionsvietnamese{
  \renewcommand{\refname}{References}
}

% Add page numbering
\pagenumbering{arabic}
\setcounter{page}{1}

\begin{document}
\title{SchubertPy: A Python Package for Schubert Calculus and Its Applications in Algebraic Geometry}
\date{}

\author{\IEEEauthorblockN{
Đặng Tuấn Hiệp\IEEEauthorrefmark{1},
Trần Duy Thanh\IEEEauthorrefmark{2}.
}
%\\
\IEEEauthorblockA{\IEEEauthorrefmark{1}
Faculty of Mathematics and Informatics, Dalat University, DLU, Lâm Đồng, Việt Nam, fbtranduythanh@gmail.com.}
\IEEEauthorblockA{\IEEEauthorrefmark{2}
Faculty of Mathematics and Informatics, Dalat University, DLU, Lâm Đồng, Việt Nam, hiepdt@dlu.edu.vn.}
}
\maketitle

\begin{abstract}
Bài viết này khám phá hành trình phát triển của \textit{phép tính Schubert} – từ những khởi đầu trực giác vào thế kỷ 19 đến nền tảng toán học chặt chẽ và các mở rộng hiện đại. Chúng tôi sẽ trình bày các bài toán đếm hình học (hình học liệt kê) đã thúc đẩy sự ra đời của phép tính Schubert, những ý tưởng độc đáo của Hermann Schubert, thách thức đặt ra bởi bài toán Hilbert thứ 15 về việc hệ thống hóa lý thuyết này, và các tiến bộ sau đó đã củng cố phép tính Schubert như một nền tảng quan trọng của hình học đại số. Câu chuyện được trình bày như một quá trình cải thiện liên tục về mặt ý tưởng và độ chặt chẽ, với việc giới thiệu các khái niệm then chốt và kỹ thuật tính toán ngày càng tinh vi. Bên cạnh phần lý thuyết, chúng tôi cũng giới thiệu \textit{SchubertPy} – một thư viện Python cho phép tính Schubert – nhằm minh họa cách thức tính toán các đối tượng của phép tính Schubert bằng công cụ máy tính.
\end{abstract}

\section{Giới thiệu}
Hình học luôn gắn liền với việc nghiên cứu các đối tượng hình dạng, kích thước, vị trí tương đối và các tính chất của không gian. Một khía cạnh cơ bản của nghiên cứu này là khả năng \textit{đếm} các đối tượng hình học thỏa mãn những điều kiện cho trước. Chẳng hạn, một câu hỏi kinh điển: 

\epigraph{\textit{``Có bao nhiêu đường thẳng trong không gian xạ ảnh ba chiều $\mathbb{P}^3$ cắt bốn đường thẳng cho trước ở vị trí tổng quát?''}} 

Câu hỏi tưởng chừng đơn giản này thực ra có đáp số không tầm thường: chính xác là \textit{hai} đường thẳng như vậy . Những bài toán tương tự xuất hiện tự nhiên trong nhiều ngành toán học khác nhau và được gọi chung là các bài toán của \textit{hình học liệt kê} (enumerative geometry)---ngành nghiên cứu về đếm số lượng các đối tượng hình học thỏa mãn các điều kiện cho trước. Các vấn đề đếm này không chỉ có ý nghĩa thuần túy về toán học mà còn có liên hệ sâu sắc đến vật lý (ví dụ như lý thuyết dây) và khoa học máy tính.

Vào cuối thế kỷ 19, nhà toán học Hermann Schubert đã tiên phong xây dựng một tập hợp phương pháp để giải quyết nhiều bài toán hình học liệt kê phức tạp, đặc biệt là những bài toán liên quan đến các không gian con tuyến tính trong không gian xạ ảnh. Công trình của ông, tập hợp trong cuốn sách kinh điển \textit{Kalkül der abzählenden Geometrie} xuất bản năm 1879, đã đặt nền móng cho lĩnh vực ngày nay gọi là \textit{phép tính Schubert} \cite{schubert1879kalkul}. Phương pháp của Schubert tỏ ra cực kỳ hiệu quả trong việc đưa ra đáp số đúng cho nhiều bài toán hóc búa. Ông đã giới thiệu các ký hiệu hình học để biểu diễn các điều kiện vị trí và xây dựng những quy tắc biến đổi các ký hiệu đó nhằm thu được kết quả đếm số lượng. Tuy nhiên, theo tiêu chuẩn chặt chẽ của toán học vào đầu thế kỷ 20, các lập luận của Schubert chưa có nền tảng vững chắc về mặt lý thuyết. Điều này đã dẫn tới việc David Hilbert đưa ra bài toán thứ 15 trong danh sách 23 bài toán nổi tiếng của ông, yêu cầu xây dựng cơ sở toán học nghiêm ngặt cho phép tính liệt kê của Schubert \cite{Encyclo_schubert_calculus}.

Trong suốt thế kỷ 20, các nhà toán học đã nỗ lực giải quyết thách thức của Hilbert. Những công cụ hiện đại từ tô-pô đại số và lý thuyết giao (intersection theory) đã được phát triển để đặt phương pháp của Schubert trên nền tảng vững chắc \cite{Kontsevich_1994}. Kết quả là phép tính Schubert đã được xây dựng lại một cách chặt chẽ thông qua việc hiểu các phép đếm như những tích trong \textit{vành đồng điều} (cohomology) của các đa tạp cờ (ví dụ như các đa tạp Grassmann) \cite{BERTRAM1999728}. Ngày nay, phép tính Schubert không chỉ dừng lại ở các kết quả cổ điển của thế kỷ 19 mà còn mở rộng đáng kể phạm vi. Những phát triển hiện đại đã đưa đến \textit{phép tính Schubert lượng tử} (quantum Schubert calculus), phép tính Schubert trên $K$-theory, và các phiên bản equivariant (đối xứng hóa) \cite{ALCO_2018__1_3_327_0}. Những mở rộng này cho phép giải quyết các bài toán trước đây không thể tiếp cận được, đồng thời tiết lộ các kết nối sâu sắc giữa hình học liệt kê với những lĩnh vực khác như đại số tổ hợp, lý thuyết biểu diễn và vật lý lý thuyết \cite{Encyclo_schubert_calculus}.

Mặc dù đã đạt được những tiến bộ lý thuyết to lớn, việc tính toán trong phép tính Schubert vẫn là một thách thức không nhỏ. Với các không gian có độ phức tạp cao, việc tự tính toán thủ công các phép nhân lớp Schubert hay các hệ số Littlewood-Richardson dễ dẫn đến sai sót và mất nhiều thời gian \cite{graysonschubert2}. Do đó, nhu cầu về các \textit{công cụ tính toán} chuyên dụng cho phép tính Schubert là rất lớn. Hiện nay, một số phần mềm toán học như SageMath đã cung cấp một số chức năng liên quan đến đa thức Schubert \cite{sagemath}, tuy nhiên chúng chưa được thiết kế chuyên sâu cho các nhà nghiên cứu trong lĩnh vực này. Để đáp ứng nhu cầu đó, chúng tôi phát triển \textit{SchubertPy}, một gói phần mềm Python cung cấp các công cụ toàn diện để tính toán và thao tác với các đa thức và lớp Schubert. SchubertPy được lấy cảm hứng từ gói Maple có tên \textit{qcalc} của Anders Skovsted Buch \cite{buch2008qcalc}, nhưng được xây dựng trên nền tảng Python hiện đại nhằm tận dụng sự linh hoạt và sức mạnh của hệ sinh thái Python (bao gồm SymPy, SageMath, NumPy, v.v.).

Trong các phần tiếp theo của bài viết, chúng tôi sẽ lần lượt giới thiệu cơ sở lý thuyết của phép tính Schubert (bao gồm các khái niệm về đa tạp Grassmann, lớp Schubert và vành đồng điều), trình bày các quy tắc tính toán chính (như quy tắc Pieri, công thức Giambelli, quy tắc Littlewood–Richardson) kèm theo các ví dụ minh họa, và sau đó giới thiệu cách sử dụng công cụ SchubertPy để giải quyết các bài toán mẫu. Cuối cùng, chúng tôi thảo luận kết luận và đề xuất một số hướng phát triển mở rộng cho cả lý thuyết và công cụ này.

\section{Cơ sở lý thuyết của phép tính Schubert}

\subsection{Đa tạp Grassmann và các đa tạp cờ}
Để giải quyết các bài toán đếm hình học một cách có hệ thống, ta cần mô tả không gian của tất cả các đối tượng hình học cần đếm dưới dạng một \textit{đa tạp đại số}. Trong các bài toán của Schubert, đối tượng thường gặp nhất là các không gian con tuyến tính trong một không gian vector cố định—gọi là \textit{không gian nền} (ambient space), thường ký hiệu là $\mathbb{C}^n$ hoặc $\mathbb{R}^n$. Tuỳ vào mục tiêu nghiên cứu, ta có thể xây dựng đa tạp Grassmann $Gr(k,n)$ dưới hai dạng: $Gr(k, \mathbb{R}^n)$ khi không gian nền là thực, hoặc $Gr(k, \mathbb{C}^n)$ khi làm việc trong ngữ cảnh phức. Trong hình học Schubert và hình học đại số, ta thường dùng $\mathbb{C}^n$ vì không gian này cho phép áp dụng các công cụ đại số mạnh như định lý Bézout, tính đồng điều và phân rã theo phân hoạch.Trong phạm vi bài viết này, ta giả định không gian nền là $\mathbb{C}^n$.

\subsubsection{\textbf{Grassmannian---Không gian của các không gian con}}

Đa tạp Grassmann $Gr(k,n)$ là không gian tham số hóa tất cả các không gian con tuyến tính $k$-chiều trong $\mathbb{C}^n$:
$$
Gr(k,n) = \left\{ V \subset \mathbb{C}^n \mid \dim V = k \right\}. 
$$

Mỗi điểm trong $Gr(k,n)$ không phải là một vector cụ thể mà là một \textit{không gian con} $k$-chiều. Đây là bước nhảy tư duy quan trọng: từ việc xét một đối tượng cụ thể, ta chuyển sang xét toàn bộ tập hợp của các không gian con có cùng chiều.

\textbf{Ví dụ}: $Gr(2,4)$ là không gian của tất cả các mặt phẳng hai chiều trong $\mathbb{C}^4$. Trong hình học xạ ảnh, điều này tương đương với tập hợp các đường thẳng trong $\mathbb{P}^3$.

Việc gom các không gian con vào trong một đa tạp chung như $Gr(k,n)$ có ý nghĩa toán học sâu sắc: nó cho phép ta áp dụng các công cụ hình học và đại số để xử lý các bài toán đếm mà trước đây vốn mang tính hình dung trực tiếp.

\subsubsection{\textbf{Từ hình học đến đại số thông qua giao cắt}}

Thay vì đếm từng trường hợp cụ thể, ta xem mỗi điều kiện hình học (ví dụ: “đi qua một điểm”, “nằm trong mặt phẳng”, “cắt một đường thẳng”) là một tập con hình học trong $Gr(k,n)$. Khi đó, bài toán đếm trở thành bài toán giao cắt các tập con này. Nếu giao cắt cho ra số hữu hạn điểm, số lượng các điểm đó chính là nghiệm cần tìm.

\textit{Ví dụ}: Trong $Gr(2,4)$, bài toán “có bao nhiêu đường thẳng cắt bốn đường tổng quát trong $\mathbb{P}^3$?” trở thành bài toán tìm giao của bốn tập con hình học trong $Gr(2,4)$, mỗi tập tương ứng với một điều kiện “cắt một đường”.

\subsubsection{\textbf{Cờ toàn phần: Hệ trục chuẩn để mô tả vị trí}}

Để mô tả một cách hình học và đại số chính xác các điều kiện như "đi qua điểm", "nằm trong mặt phẳng", ta cần một hệ quy chiếu hình học – đó là \textit{cờ toàn phần} $F_\bullet$:
$$
0 \subset F_1 \subset F_2 \subset \cdots \subset F_n = \mathbb{C}^n, \quad \dim F_i = i. 
$$
Cờ chuẩn $F_\bullet$ chia không gian $\mathbb{R}^n$ thành một chuỗi các lớp con theo thứ bậc. Cho một không gian con $P \in \mathrm{Gr}(k,n)$ (tức là một $k$-mặt phẳng qua gốc), ta quan tâm đến vị trí tương đối giữa $P$ và cờ $F_\bullet$. Mỗi điều kiện hình học có thể được mô tả bằng cách $P$ giao cắt các lớp $F_j$ như sau:
$$
\dim(P \cap F_{n - k + i - \lambda_i}) \geq i, \quad \text{với } i = 1,\dots,k.
$$
Trong đó $\lambda = (\lambda_1, \dots, \lambda_k)$ là một phân hoạch mô tả mức độ khắt khe của yêu cầu giao cắt ($\lambda_i$ càng lớn thì điều kiện càng chặt).

Xét về trực giác hình học, ta cố định một cờ chuẩn $F_\bullet$, và để không gian con $P \in \mathrm{Gr}(k,n)$ thay đổi trong toàn bộ Grassmannian. Tập hợp tất cả các $P$ sao cho $P$ cắt các tầng $F_j$ của cờ $F_\bullet$ với kiểu cắt xác định bởi một phân hoạch $\lambda = (\lambda_1, \dots, \lambda_k)$, tức là thỏa:
$$
\dim(P \cap F_{n - k + i - \lambda_i}) = i \quad \text{với } i = 1,\dots,k,
$$
chính là một Schubert cell $X^\circ_\lambda \subset \mathrm{Gr}(k,n)$.

Điều kiện trên không đếm số "vị trí cắt", mà yêu cầu phần giao giữa $P$ và các lớp con $F_j$ của cờ phải có đúng $i$ chiều tại từng bước. Mỗi Schubert cell tương ứng với một kiểu cắt chính xác, và tập hợp các cell này phân hoạch Grassmannian thành các quỹ đạo Bruhat rời rạc.

\subsubsection{\textbf{Schubert variety $\Omega_\lambda$}}
Với mỗi phân hoạch 
$$\lambda = (\lambda_1, \dots, \lambda_k)$$
và một cờ toàn phần $F_\bullet$, ta định nghĩa \textit{Schubert variety}:
$$
\Omega_\lambda(F_\bullet) = \left\{ P \in Gr(k,n) \mid \dim(P \cap F_{n-k+i - \lambda_i}) \geq i\right\}
$$
với $i = 1, \dots, k$.

Đây là tập hợp các không gian con $k$-chiều thỏa mãn các điều kiện vị trí được mã hóa bởi $\lambda$. $\Omega_\lambda$ là một đa tạp con đại số trong $Gr(k,n)$, thường là bất khả quy và có codimension bằng tổng các phần tử của $\lambda$.

\textbf{Ví dụ}: Xét $Gr(2,4)$, tức không gian của các mặt phẳng trong $\mathbb{C}^4$. Giả sử ta chọn cờ chuẩn:
\begin{align*}
F_1 &= \langle e_1 \rangle, \\
F_2 &= \langle e_1, e_2 \rangle, \\
F_3 &= \langle e_1, e_2, e_3 \rangle, \\
F_4 &= \mathbb{C}^4.
\end{align*}

Xét mặt phẳng $P = \text{span}(e_2 + e_3,\ e_3 + e_4)$. Ta tính các giao với cờ:
\begin{align*}
\dim(P \cap F_1) &= 0, \\
\dim(P \cap F_2) &= 1, \\
\dim(P \cap F_3) &= 2, \\
\dim(P \cap F_4) &= 2.
\end{align*}

Dựa vào công thức điều kiện của $\Omega_\lambda$, ta thấy $P$ thỏa mãn:
\[ \dim(P \cap F_3) \geq 1, \quad \dim(P \cap F_2) \geq 0. \]
Tức là $P \in \Omega_{(1,0)}(F_\bullet)$. Ngoài ra, ta cũng có thể kiểm tra liệu $P$ có thuộc các Schubert variety khác như $\Omega_{(2,1)}(F_\bullet)$ hay không. Điều kiện tương ứng là:
\[ \dim(P \cap F_1) \geq 1, \quad \dim(P \cap F_2) \geq 2. \]
Tuy nhiên, trong ví dụ này ta có:
\[ \dim(P \cap F_1) = 0, \quad \dim(P \cap F_2) = 1, \]
nên $P \notin \Omega_{(2,1)}(F_\bullet)$. Điều này cho thấy các phân hoạch với tổng số ô lớn hơn sẽ mô tả các điều kiện vị trí chặt hơn và tập các $P$ thỏa mãn sẽ nhỏ hơn.


Biểu đồ Young tương ứng với $\lambda = (1,0)$ gồm một ô ở hàng đầu tiên, biểu thị một điều kiện cắt bổ sung. Tổng số ô là 1, nên codimension của $\Omega_{(1,0)}$ là 1 trong $Gr(2,4)$.

Khi ta lấy giao của nhiều $\Omega_\lambda$ khác nhau (với các cờ tổng quát), ta mô tả tập các $P$ thỏa đồng thời nhiều điều kiện vị trí. Nếu tổng các codimension đúng bằng $\dim Gr(k,n)$, giao sẽ là tập hữu hạn các điểm.

\textbf{Ví dụ mở rộng}: Trong $Gr(2,4)$, giao của bốn $\Omega_{(1,0)}$ tương ứng với bài toán đếm số đường thẳng cắt bốn đường tổng quát trong $\mathbb{P}^3$. Kết quả là đúng hai điểm rời rạc – tương ứng với hai đường thỏa mãn tất cả các điều kiện.

\subsubsection{\textbf{Kết nối với đồng điều}}
Đa tạp Schubert là cầu nối giữa hình học đại số và lý thuyết đồng điều: mỗi điều kiện hình học trở thành một lớp trong vành đồng điều. Khi tính giao các Schubert variety, ta thực chất đang nhân các lớp đại số tương ứng. Việc đếm số giao chính là bài toán tổ hợp trong ngôn ngữ của đại số – đây là điểm mạnh của phép tính Schubert, sẽ được trình bày chi tiết hơn trong phần tiếp theo.



\subsection{Lớp Schubert và vành đồng điều của Grassmann}
Vành đồng điều của Grassmannian $Gr(k,n)$ có dạng:
$$
H^\star(Gr(k,n)) = \bigoplus_{\lambda \subseteq k \times (n-k)} \mathbb{Z} \cdot \sigma_\lambda.
$$
Khi đó, tích của hai lớp Schubert được định nghĩa:
$$
\sigma_{\lambda} \cdot \sigma_{\mu} = \sum_{\nu} c_{\lambda, \mu}^{\nu} \, \sigma_{\nu}.
$$
Mỗi đa tạp Schubert $\Omega_\lambda(F_\bullet)$ trong $Gr(k,n)$ xác định một \textit{lớp đồng điều} (cohomology class) $[\Omega_\lambda]$ trong vành đồng điều $H^*(Gr(k,n), \mathbb{Z})$ của đa tạp Grassmann tương ứng. Các lớp này không phụ thuộc vào lựa chọn cụ thể của cờ $F_\bullet$ (vì nếu thay đổi cờ ở vị trí tổng quát thì lớp đồng điều thu được vẫn giống nhau). Tập hợp tất cả các \textit{lớp Schubert} $\{\sigma_\lambda = [\Omega_\lambda]\}$, ứng với mọi partition $\lambda$ khả dĩ, tạo thành một cơ sở $\mathbb{Z}$-tự do của $H^*(Gr(k,n),\mathbb{Z})$. Nói cách khác, bất kỳ lớp đồng điều nào trên $Gr(k,n)$ cũng có thể biểu diễn duy nhất dưới dạng tổ hợp tuyến tính (hữu hạn) các lớp Schubert.

Cấu trúc của vành đồng điều $H^*(Gr(k,n),\mathbb{Z})$ rất đặc biệt: nó là một vành giao hoán có đơn vị, trong đó phép nhân hai lớp đồng điều $[\Omega_{\lambda}] \cdot [\Omega_{\mu}]$ cũng chính là phép tính giao của hai đa tạp Schubert tương ứng trong $Gr(k,n)$. Kết quả của phép nhân là một lớp đồng điều mới, có thể biểu diễn trên cơ sở Schubert:
\[ [\Omega_{\lambda}] \cdot [\Omega_{\mu}] \;=\; \sum_{\nu} c_{\lambda,\mu}^{\,\nu}\, [\Omega_{\nu}], \]
trong đó các hệ số $c_{\lambda,\mu}^{\,\nu}$ là các số nguyên không âm gọi là các \textit{hệ số Littlewood-Richardson}. Mỗi hệ số $c_{\lambda,\mu}^{\,\nu}$ về mặt hình học biểu thị số lượng (đếm theo bội số thích hợp) các điểm giao của ba đa tạp Schubert $\Omega_{\lambda}, \Omega_{\mu}$ và $\Omega_{\nu^c}$ trong $Gr(k,n)$, với $\nu^c$ là phân hoạch bổ sung đảm bảo tổng các điều kiện bằng đúng $\dim Gr(k,n)$. Khi $\lambda, \mu, \nu$ thỏa mãn điều kiện tổng các phần bằng $k(n-k)$, khi đó $\Omega_{\lambda} \cap \Omega_{\mu} \cap \Omega_{\nu}$ gồm một số hữu hạn điểm, và số điểm này chính là $c_{\lambda,\mu}^{\,\nu}$. Do đó, việc tìm số nghiệm của bài toán hình học liệt kê ban đầu tương đương với việc xác định một hệ số $c_{\lambda,\mu}^{\,\nu}$ tương ứng trong phép nhân các lớp Schubert.

Bằng cách chuyển bài toán hình học về ngôn ngữ của vành đồng điều, ta đã đưa vấn đề đếm hình học về một bài toán đại số tổ hợp: tính các hệ số xuất hiện khi nhân các phần tử cơ sở (các lớp Schubert) trong một vành giao hoán. Bài toán này, thường được gọi là \textit{bài toán đặc trưng} (characteristic problem) theo cách gọi của Schubert, chính là trọng tâm của phép tính Schubert. Trong phần tiếp theo, chúng ta sẽ điểm qua những công cụ cổ điển để giải quyết bài toán này, bao gồm các công thức và quy tắc mang tên Pieri, Giambelli và Littlewood-Richardson.

\section{Các quy tắc tính toán trong phép tính Schubert cổ điển}

\subsection{Quy tắc Pieri}
Một trường hợp đặc biệt quan trọng của bài toán nhân các lớp Schubert là nhân một lớp Schubert bất kỳ với một lớp đặc biệt biểu diễn bằng một phân hoạch có dạng một hàng hoặc một cột. Những lớp tương ứng thường được gọi là \textit{các lớp đặc biệt} (special Schubert classes), ví dụ $\sigma_{(a)}$ (một hàng dài $a$ ô) hoặc $\sigma_{(1^a)}$ (một cột cao $a$ ô) trong $H^*(Gr(k,n))$. \textit{Công thức Pieri} cung cấp cách tính đơn giản cho trường hợp này. Phát biểu đơn giản của quy tắc Pieri:
\[ \sigma_{\lambda} \cdot \sigma_{(a)} \;=\; \sum_{\nu} \sigma_{\nu}, \]
trong đó tổng lấy qua tất cả các phân hoạch $\nu$ thu được bằng cách thêm $a$ ô không nằm cùng một cột vào biểu đồ (diagram) của $\lambda$. Tương tự,
\[ \sigma_{\lambda} \cdot \sigma_{(1^a)} \;=\; \sum_{\nu} \sigma_{\nu}, \]
trong đó $\nu$ thu được bằng cách thêm $a$ ô không nằm cùng một hàng vào biểu đồ của $\lambda$. Những phân hoạch $\nu$ thu được theo cách trên đảm bảo $\nu$ vẫn phù hợp với $Gr(k,n)$ (không vượt quá kích thước lưới $k \times (n-k)$ của biểu đồ Young). Công thức Pieri thực chất cho ta một quy tắc tổ hợp để liệt kê các lớp xuất hiện khi nhân với một lớp đặc biệt, và tất cả các hệ số trong hai biểu thức trên đều bằng 1. Mặc dù đơn giản, công thức Pieri đủ mạnh để giải quyết nhiều bài toán đếm, đặc biệt là những bài toán trong đó một trong các điều kiện tương ứng với một phân hoạch dạng hàng hoặc cột (ví dụ bài toán "có bao nhiêu đường thẳng đi qua một điểm cố định và thỏa các điều kiện khác..." v.v.).

\subsection{Công thức Giambelli}
Pieri giải quyết trường hợp nhân với lớp đặc biệt. Để biểu diễn một lớp Schubert bất kỳ thành phần tử của vành đa thức sinh bởi các lớp đặc biệt, \textit{công thức Giambelli} cung cấp mối liên hệ cần thiết. Công thức Giambelli phát biểu rằng mỗi lớp Schubert $\sigma_{\lambda}$ trong $H^*(Gr(k,n))$ có thể biểu diễn như một định thức (determinant) với các phần tử là các lớp đặc biệt. Cụ thể, nếu 
$$\lambda = (\lambda_1,\lambda_2,\dots,\lambda_r)$$ 
có $r$ phần, khi đó:
$$ \sigma_{\lambda} = \det\big[\sigma_{(\lambda_i + j - i)}\big]_{1 \le i,j \le r}.
$$
Ở đây vế phải là định thức của ma trận $r \times r$ với phần tử hàng $i$, cột $j$ là lớp đặc biệt $\sigma_{(\lambda_i + j - i)}$. Ví dụ, 
$$\sigma_{(a,b)} = \sigma_{(a)}\cdot \sigma_{(b-1)} - \sigma_{(a-1)}\cdot \sigma_{(b)}$$ 
với $a \ge b$ là một trường hợp cụ thể của công thức trên. Công thức Giambelli cho phép ta tạo một cầu nối giữa các lớp Schubert tổng quát và các lớp đặc biệt, từ đó mọi phép tính trên các lớp Schubert có thể quy về tính toán với các lớp đặc biệt (mà ta đã có quy tắc Pieri để xử lý).

\subsection{Quy tắc Littlewood-Richardson}
Trường hợp tổng quát nhất---nhân hai lớp Schubert bất kỳ với nhau $\sigma_{\lambda} \cdot \sigma_{\mu}$---được giải quyết bằng \textit{quy tắc Littlewood-Richardson (L--R)}. Quy tắc L--R là một kết quả cơ bản trong đại số tổ hợp và lý thuyết biểu diễn, cung cấp một cách đếm thuần túy tổ hợp để xác định các hệ số $c_{\lambda,\mu}^{\,\nu}$ trong tích Schubert. Phát biểu của quy tắc L--R sử dụng ngôn ngữ của \textit{biểu đồ Young} và \textit{bảng Young} (Young tableau). Ý tưởng chính như sau: một \textit{bảng Young nửa chuẩn tắc (semi-standard Young tableau, SSYT)} của dạng hình (shape) $\nu/\lambda$ (biểu đồ của $\nu$ bỏ đi biểu đồ của $\lambda$) với \textit{nội dung} (content) $\mu$ được định nghĩa là cách điền các số tự nhiên vào các ô của hình $\nu/\lambda$ sao cho các hàng không giảm từ trái sang phải và các cột tăng từ trên xuống dưới, và số lượng các ô chứa số $i$ đúng bằng $\mu_i$ với $\mu = (\mu_1,\mu_2,\dots)$. Một bảng như vậy được gọi là \textit{bảng Littlewood-Richardson} nếu thỏa thêm điều kiện đặc biệt gọi là \textit{điều kiện Littlewood-Richardson}: khi đọc các ô theo thứ tự từ phải sang trái, từ dòng trên xuống dòng dưới (được gọi là \textit{từ LR}), tại mọi điểm của quá trình đọc, số lần xuất hiện của $j$ không bao giờ vượt quá số lần xuất hiện của $j+1$ đối với mọi $j$. Số lượng các bảng Littlewood-Richardson dạng $\nu/\lambda$ với nội dung $\mu$ chính là hệ số $c_{\lambda,\mu}^{\,\nu}$.

Quy tắc Littlewood-Richardson tuy trừu tượng hơn, nhưng cung cấp một công cụ cực kỳ mạnh để tính toán trong vành đồng điều của $Gr(k,n)$ và giải các bài toán hình học liệt kê. Từ quy tắc này, ta có thể suy ra cả công thức Pieri (là trường hợp đặc biệt khi $\mu$ là một hàng hoặc một cột duy nhất) lẫn công thức Giambelli (khi kết hợp với tính đối ngẫu Poincaré trong vành đồng điều). Trong thực tế, việc áp dụng quy tắc L-R có thể khá phức tạp vì số lượng bảng Young cần xét có thể rất nhiều khi kích thước các phân hoạch tăng. Tuy nhiên, nó đảm bảo rằng về mặt lý thuyết, mọi phép nhân Schubert đều có thể giải được một cách triệt để.

Tóm lại, ba kết quả kinh điển---Pieri, Giambelli và Littlewood-Richardson---tạo thành nền tảng tính toán cho phép tính Schubert cổ điển. Chúng cho phép chuyển mọi bài toán đếm hình học tuyến tính cổ điển (trên các Grassmannian) về một bài toán tổ hợp về biểu đồ Young. Những công cụ này được phát triển trong nửa đầu thế kỷ 20 (Pieri 1890, Giambelli 1902, Littlewood-Richardson 1934) đã giúp hoàn thành phần lớn chương trình đặt cơ sở chặt chẽ cho phép tính Schubert như Hilbert đề ra, kết hợp với sự phát triển của lý thuyết đồng điều và hình học đại số hiện đại (các kết quả của Ehresmann, Chow, Chevalley và các nhà toán học khác).


\section{Phép tính Schubert lượng tử}
\subsection{Mở rộng phép tính Schubert}
Đến thập niên 1990, dưới sự ảnh hưởng của vật lý lý thuyết (đặc biệt là lý thuyết dây), khái niệm \textit{phép tính Schubert lượng tử} ra đời như một mở rộng hiện đại của phép tính Schubert. Phép tính Schubert lượng tử đưa thêm vào bức tranh cổ điển thông tin về các \textit{đường cong đại số} (đặc biệt là các đường cong ``hữu tỉ'' – tương đương hình học với $\mathbb{CP}^1$) bên trong không gian. Thông qua lý thuyết \textit{đồng điều lượng tử}, người ta định nghĩa một phép nhân mới (gọi là \textit{phép nhân lượng tử}) trên các lớp Schubert sao cho ngoài các giao cắt thông thường, còn tính đến khả năng các lớp này \textit{giao nhau} khi được \textit{nối với nhau bởi một hay nhiều đường cong}. Nói cách khác, nếu trong đồng điều cổ điển hai lớp Schubert $\sigma_u$ và $\sigma_v$ chỉ giao phiếm hàm khi chúng thực sự cắt nhau, thì trong đồng điều lượng tử, $\sigma_u$ và $\sigma_v$ còn có thể cho kết quả khác $0$ nếu tồn tại đường cong (ví dụ: một đường cong hữu tỉ) nằm trong không gian xét, nối liền các vị trí mà hai lớp này ánh xạ tới. Hệ số cho những \textit{giao điểm lượng tử} này chính là các \textit{bất biến Gromov–Witten}---về cơ bản là số lượng (đếm có hướng hoặc đếm xấp xỉ) các đường cong thỏa mãn những điều kiện giao hình học tương ứng. Kết quả, phép tính Schubert lượng tử thiết lập một cấu trúc đại số mới trên không gian đồng điều: \textit{vành đồng điều lượng tử (nhỏ)} $QH^*(X)$, trong đó $X$ là không gian xét (ví dụ $X$ có thể là đa tạp cờ hoặc Grassmann). Vành này mở rộng vành đồng điều thông thường bằng cách thêm các biến hình thức (thường ký hiệu $q$) mang thông tin về độ của đường cong, và phép nhân chén thông thường được \textit{biến dạng} (deform) thành phép nhân lượng tử có kèm các hệ số $q$.

Vành đồng điều lượng tử của Grassmannian $Gr(k,n)$ có dạng:
$$
QH^\star(Gr(k,n)) = \bigoplus_{\lambda \subseteq k \times (n-k)} \mathbb{Z}[q] \cdot \sigma_\lambda,
$$
với $q$ là biến lượng tử có bậc $n$ (tương ứng với số điều kiện cần để có đường cong bậc 1 trong không gian đó). Trong $QH^*(X)$ người ta định nghĩa tích lượng tử:
$$
\sigma_u * \sigma_v = \sum_{d \geq 0} \sum_w N^{w,d}_{u,v} \, q^d \, \sigma_w,
$$
trong đó
\begin{itemize}
    \item $\sigma_u, \sigma_v, \sigma_w$ là các lớp Schubert (hay nói chung là các lớp đồng điều) trong không gian $X$. 
    \item Hệ số $N^{w,d}_{u,v}$ là các bất biến Gromov–Witten 3 điểm (genus 0) của $X$, trực tiếp \textit{đếm số lượng đường cong hữu tỉ cấp $d$} trong $X$ đồng thời thỏa mãn ba điều kiện: đi qua vị trí tổng quát thuộc các lớp Schubert $\sigma_u, \sigma_v$ và $\sigma_w$ (chỉ định thông qua tính chất giao với cờ cố định). 
\end{itemize}
Khi $d=0$ (đường cong cấp 0 có thể hiểu là giao điểm thông thường), $N^{w,0}_{u,v}$ chính là số giao điểm hữu hạn thông thường – trùng với hằng số cấu trúc cổ điển. Bởi vậy, nếu bỏ qua các bội $q^d$ với $d>0$, công thức trên thu về phép nhân Schubert cổ điển. Ngược lại, với $d>0$, $N^{w,d}_{u,v}$ cho ta thông tin về \textit{bài toán đếm đường cong}: ví dụ, trong không gian cờ cỡ nhỏ, một số bất biến Gromov–Witten quen thuộc đếm số đường conic hoặc đường cong bậc cao hơn thỏa mãn các điều kiện hình học nhất định. Các \textit{đa thức Schubert lượng tử} được xây dựng để biểu diễn các lớp Schubert trong vành $QH^*(X)$ cũng như tính toán hiệu quả phép nhân lượng tử. Chẳng hạn, trong trường hợp $X$ là đa tạp Grassmann, các đa thức Schubert lượng tử (do Fomin, Gelfand và Postnikov đề xuất) mở rộng đa thức Schubert cổ điển bằng cách thêm các biến $q$ và thỏa mãn các quan hệ mới phản ánh các đếm đường cong (ví dụ: các công thức \textit{Pieri lượng tử} và \textit{Giambelli lượng tử}). Điều này cho phép ta tính toán nhanh chóng phép nhân lượng tử $\sigma_u * \sigma_v$ bằng cách nhân các đa thức tương ứng rồi ánh xạ kết quả theo các quan hệ lượng tử (ví dụ: với Grassmann $Gr(k,n)$, quan hệ cơ bản thường có dạng $h^n = q$ với $h$ là lớp hyperplane thích hợp) \cite{buch2001quantumcohomologygrassmannians}. 

\subsection{Vai trò trong toán học}
Phép tính Schubert lượng tử không chỉ nối tiếp truyền thống \textit{hình học liệt kê} của Schubert mà còn mở ra các hướng nghiên cứu mới trong toán học hiện đại. Trước hết, nó cung cấp một công cụ mạnh để giải quyết các bài toán đếm hình học phức tạp từng nằm ngoài tầm với của phương pháp cổ điển. Ví dụ, bằng cách sử dụng đồng điều lượng tử, các nhà toán học đã tính được số đường cong bậc $d$ đi qua các vị trí cho trước trên nhiều đa tạp (một bài toán có liên hệ tới giả thuyết đường cong cong của vật lý). Kết quả tiên đoán bởi vật lý về số lượng đường cong trên các đa tạp đặc biệt (như đa tạp Calabi–Yau) đã được xác nhận thông qua tính toán Gromov–Witten, cho thấy sức mạnh dự đoán của phép tính Schubert lượng tử. Hơn nữa, cấu trúc đại số của $QH^*(X)$ (một \textit{vành giao hoán có đơn vị} với phép nhân biến dạng) mang tính chất của một \textit{đại số Frobenius}, gắn liền với hiện tượng đối ngẫu gương và các hệ phương trình vi phân khả tích (phương trình $q$-hệ số WDVV) trong hình học. Trong một số trường hợp đặc biệt, đồng điều lượng tử của đa tạp đồng nhất trùng với các cấu trúc nổi tiếng khác trong toán học: chẳng hạn Edward Witten đã chỉ ra rằng $QH^*(Gr(k,n))$ tương đương với \textit{đại số Verlinde} mô tả phép tính fusion trong lý thuyết nhóm con hữu hạn \cite{witten1993verlindealgebracohomologygrassmannian}. Điều này giải quyết một cách hình học bài toán đếm của lý thuyết biểu diễn (tính kích thước không gian các hàm suy rộng) và cũng đồng thời giải thích một hiện tượng vật lý 2 chiều (tương ứng với mô hình trường $WZW$ trên mặt cầu) bằng ngôn ngữ hình học đại số. Một ứng dụng nổi bật khác là lời giải cho \textit{bài toán phổ điểm riêng} trong đại số tuyến tính: thông qua phép tính Schubert lượng tử trên $Gr(k,n)$, Agnihotri và Woodward đã mô tả đầy đủ các bất đẳng thức đặc trưng cho khả năng tồn tại ma trận đơn vị với phổ cho trước (liên quan đến \textit{dự đoán Horn} về cộng các phổ ma trận) \cite{agnihotri1997eigenvaluesproductsunitarymatrices}. Kết quả này kết nối bất ngờ giữa hình học (các bất biến Gromov–Witten trên cờ) và đại số tuyến tính, đồng thời khẳng định tính \textit{dương} của mọi bất biến Gromov–Witten (một tính chất quan trọng được dự đoán trước đó). Như vậy, phép tính Schubert lượng tử đã liên kết các lĩnh vực tưởng chừng rất xa nhau – từ hình học đại số, tổ hợp cho đến lý thuyết biểu diễn và đại số tuyến tính – tạo nên một bức tranh đa ngành phong phú.

\subsection{Vai trò trong vật lý lý thuyết}
Phép tính Schubert lượng tử (và lý thuyết đồng điều lượng tử nói chung) có nguồn gốc sâu xa từ vật lý lý thuyết, cụ thể là từ \textit{lý thuyết trường lượng tử topo} và \textit{lý thuyết dây}. Các bất biến Gromov–Witten ban đầu xuất hiện như các \textit{biên độ xác suất} trong lý thuyết topological sigma-model A (mô tả các mặt cầu holomorphic ánh xạ vào một không gian đích $X$). Việc \textit{đếm} các đường cong chỉnh hình (holomorphic) này trong vật lý tương đương với việc tính các tích phân trên \textit{không gian moduli} của các ánh xạ ổn định---đó chính là cách các nhà toán học định nghĩa nghiêm túc bất biến Gromov–Witten. Do đó, phép tính Schubert lượng tử là cầu nối giữa toán học và vật lý: mỗi đẳng thức trong đồng điều lượng tử (chẳng hạn $h^{n} = q$ trong $QH^*(\mathbb{CP}^{n-1})$) có thể diễn giải như một quá trình vật lý (ở đây là $n$ hạt điểm giao nhau có thể được thay thế bằng một hạt điểm duy nhất nếu có sự xuất hiện của một \textit{instanton} là một cầu phương vị---chính là đường cong $\mathbb{CP}^1$---đóng vai trò kết nối). Đặc biệt, \textit{đối xứng gương}---mối liên hệ huyền bí giữa hai không gian hình học tưởng chừng khác biệt – được kiểm chứng và nghiên cứu thông qua đồng điều lượng tử: vành đồng điều lượng tử của $X$ (A-model) tương ứng với cấu trúc giải tích phức trên \textit{gương} của $X$ (B-model), và ngược lại \cite{nlab:gromov-witten_invariants}. Nhiều bài toán đếm đường cong khó (ví dụ: đếm số đường cong trên một đa tạp Calabi–Yau 3 chiều như quintic) đã được giải quyết nhờ đối xứng gương, từ đó thúc đẩy sự phát triển của phép tính Schubert lượng tử và lý thuyết Gromov–Witten trong toán học. Ngày nay, các bất biến Gromov–Witten còn được xem là một phần của các \textit{đại số lượng tử} trong lý thuyết trường, liên hệ mật thiết với các đại lượng vật lý như \textit{hàm phân hoạch} (partition function) và \textit{hàm tương quan} của lý thuyết dây. Sự tương đồng cấu trúc giữa vành đồng điều lượng tử và những lý thuyết vật lý 2 chiều cho phép hai ngành học hỏi lẫn nhau: vật lý gợi ý nhiều giả thuyết hình học táo bạo (như đối xứng gương, giả thuyết về tính toàn vẹn Gromov–Witten), trong khi các kết quả toán học cung cấp cơ sở chặt chẽ để hiểu rõ hơn các mô hình vật lý.



\section{Các hướng mở rộng hiện đại}
\subsection{Đồng điều lượng tử hiệp biến}
Phép tính Schubert lượng tử, sau hơn ba thập kỷ phát triển, vẫn là một lĩnh vực sôi động với nhiều hướng nghiên cứu mở rộng. Một trong những hướng đó là việc tổng quát hóa phép tính Schubert lượng tử sang \textit{trường hợp hiệp biến} (equivariant quantum cohomology), nơi người ta xét đồng thời tác động của một nhóm đối xứng lên không gian. Đồng điều lượng tử hiệp biến đòi hỏi kết hợp cả lý thuyết đặc trưng Chern và tích phân trên moduli, dẫn đến các kết quả phức tạp hơn nhưng cũng giàu thông tin hơn về không gian (ví dụ: công trình tính toán $QH_T^*(G/P)$ cho các đa tạp cờ $G/P$ dưới tác động của torus). 

\subsection{Lý thuyết K}
Hướng thứ hai là mở rộng sang \textit{lý thuyết K lượng tử} và xa hơn là \textit{đồng điều elliptic lượng tử}, tức là xây dựng các phiên bản của phép tính Schubert trong lý thuyết K (đối tượng là các lớp bó vector (vector bundle)) hoặc trong lý thuyết elliptic (liên quan đến các hàm theta và dạng module). Những mở rộng này gắn liền với việc đếm không chỉ các đối tượng hình học đơn giản mà cả các bó vector holomorphic trên đường cong, và có liên hệ với các giả thuyết vật lý mới (ví dụ: đối ngẫu lượng tử – K và đối xứng gương ở mức độ K). Về mặt tổ hợp, các nhà nghiên cứu tiếp tục tìm kiếm và chứng minh các \textit{tính chất tương tự cổ điển} cho các hệ số Gromov–Witten: chẳng hạn, tính \textit{dương} và \textit{nguyên} của các hệ số trong phép nhân lượng tử (tương tự tính dương của hằng số Littlewood–Richardson cổ điển) đã được chứng minh cho nhiều trường hợp, cho thấy ý nghĩa đại số sâu sắc của những con số đếm đường cong này \cite{XXX}. 

\subsection{Các hướng khác}
Ngoài ra, phép tính Schubert lượng tử còn mở rộng sang các không gian mới ngoài phạm vi cờ cổ điển: ví dụ, tính toán đồng điều lượng tử cho các \textit{đa tạp Grassmann đẳng hướng} (loại B, C, D) và các \textit{đa tạp cờ tổng quát} (bao gồm cả các trường hợp nhóm Lie ngoại lệ) đang được tiến hành, đòi hỏi kết hợp nhiều kỹ thuật từ đại số tổ hợp đến hình học đại số hiện đại. Một hướng liên ngành khác là mối quan hệ giữa \textit{tính khả tích} (integrability) và đồng điều lượng tử: phương trình \textit{vi phân lượng tử} (xuất phát từ việc xét hàm thế Gromov–Witten) thường trùng với các hệ khả tích cổ điển (như hệ Toda), gợi mở mối liên hệ giữa phép tính Schubert lượng tử và lý thuyết soliton, hàm tau,...




\section{Kết luận}
Phép tính Schubert, từ những khởi đầu đơn giản trong hình học liệt kê, đã phát triển thành một lý thuyết toán học phong phú và sâu sắc. Các công cụ cơ bản như quy tắc Pieri, công thức Giambelli và quy tắc Littlewood-Richardson đã cung cấp nền tảng vững chắc cho việc giải quyết các bài toán đếm trong hình học đại số. Sự phát triển của phép tính Schubert lượng tử và các hướng mở rộng hiện đại đã mở ra những chân trời mới cho nghiên cứu và ứng dụng.

Trong tương lai, việc phát triển các công cụ tính toán hiệu quả, như gói phần mềm \texttt{SchubertPy}, sẽ tiếp tục đóng vai trò quan trọng trong việc thúc đẩy nghiên cứu và ứng dụng của phép tính Schubert. Các hướng nghiên cứu mới, bao gồm việc kết hợp với các lĩnh vực khác như lý thuyết biểu diễn, hình học đại số và vật lý lý thuyết, hứa hẹn sẽ mang lại nhiều kết quả thú vị và bất ngờ.

\section{Công cụ \texttt{SchubertPy} và ví dụ minh họa}
Trong phần này, chúng tôi giới thiệu \texttt{SchubertPy} – một gói Python cho phép tính Schubert được phát triển nhằm hỗ trợ tính toán và nghiên cứu. Gói \texttt{SchubertPy} cung cấp các lớp đối tượng cho các đa tạp Grassmann loại A, B, C, D (bao gồm cả các trường hợp Grassmannian cổ điển và các Grassmannian đẳng hướng trong không gian đối xứng) và các hàm để thực hiện các phép tính đồng điều và đồng điều lượng tử trên đó. Phần mềm này được xây dựng lấy cảm hứng từ gói \textit{qcalc} trên Maple của Buch \cite{buch2008qcalc}, và cũng có chức năng tương tự như một số phần trong SageMath \cite{SageMath2024}, nhưng chuyên biệt hơn cho phép tính Schubert và tối ưu cho môi trường Python. \texttt{SchubertPy} được viết bằng Python 3, mã nguồn mở trên GitHub và phân phối qua PyPI (có thể cài đặt nhanh bằng \texttt{pip install schubertpy}). Dưới đây, chúng tôi sẽ minh họa một số tính năng chính của \texttt{SchubertPy} thông qua các ví dụ tương tác.

\subsection{Các đa tạp Grassmann hỗ trợ}
Gói \texttt{SchubertPy} hiện hỗ trợ các loại Grassmannian sau:
\begin{itemize}
    \item Loại A: Grassmann thông thường $Gr(k,n)$.
    \item Loại C: Grassmann các không gian đẳng hướng trong không gian symplectic (ký hiệu $IG$).
    \item Loại B và D: Grassmann các không gian đẳng hướng trong không gian chính quy (ký hiệu $OG$ cho cả hai trường hợp).
\end{itemize}
Các đối tượng Grassmannian tương ứng có thể được tạo ra bằng các lớp Python \texttt{Grassmannian}, \texttt{IsotropicGrassmannian}, \texttt{OrthogonalGrassmannian}. Người dùng có thể xác định loại của Grassmannian bằng cách in đối tượng, tương tự như cách gói qcalc làm trên Maple. Ví dụ:

Ví dụ:
\small
\begin{verbatim}
>>> gr = Grassmannian(2, 5)
>>> print(gr)
Type A; (k,n)=(3,5); Gr(2,5); deg(q)=5

>>> ig = IsotropicGrassmannian(2, 6)
>>> print(ig)
Type C; (k,n)=(1,3); IG(2,6); deg(q)=5

>>> og = OrthogonalGrassmannian(2, 7)
>>> print(og)
Type B; (k,n)=(1,3); OG(2,7); deg(q)=4

>>> og = OrthogonalGrassmannian(2, 6)
>>> print(og)
Type D; (k,n)=(1,2); OG(2,6); deg(q)=3
\end{verbatim}
\normalsize

Trong các thông tin in ra:
- \texttt{Type A/B/C/D} cho biết loại Grassmannian.
- $(k,n)$ là các tham số của Grassmann (ví dụ $Gr(2,5)$ tương ứng với $k=2, n=5$).
- \texttt{deg(q)} là bậc của biến $q$ trong vòng đồng điều lượng tử (tức là số điều kiện cần để có đường cong độ 1 trong không gian đó).

\subsection{Lớp Schubert và các lớp đặc biệt}
Trong \texttt{SchubertPy}, các \textit{lớp Schubert} được đánh chỉ số bằng các phân hoạch (partition) phù hợp với từng loại Grassmann. Với các Grassmannian đẳng hướng (loại B, C, D), gói sử dụng phân hoạch $k$-strict (phân hoạch với một số ràng buộc bổ sung) để biểu diễn các lớp Schubert tương ứng tính đến điều kiện đẳng hướng. Người dùng có thể liệt kê các lớp Schubert và nhận diện lớp điểm (lớp có codimension lớn nhất) cũng như các lớp sinh (các lớp Schubert đặc biệt tạo sinh vành đồng điều) bằng các hàm tiện ích:
\begin{itemize}
    \item \texttt{schub\_classes()} – liệt kê toàn bộ các lớp Schubert (ký hiệu $S[\lambda]$ với $\lambda$ là phân hoạch).
    \item \texttt{point\_class()} – trả về lớp điểm (ứng với phân hoạch lớn nhất).
    \item \texttt{generators()} – trả về danh sách các lớp sinh đặc biệt của vành đồng điều (thường tương ứng với các phân hoạch dạng một ô ở các góc của lưới Young $k \times (n-k)$).
\end{itemize}

Ví dụ:
\small
\begin{verbatim}
>>> gr = Grassmannian(2,5)
>>> print(gr)
Type A; (k,n)=(3,5); Gr(2,5); deg(q)=5

>>> gr.schub_classes()
[S[],
 S[1],   S[1,1],
 S[2], S[2,1], S[2,2],
 S[3], S[3,1], S[3,2], S[3,3]]

>>> gr.point_class()
S[3,3]

>>> gr.generators()
[S[1], S[2], S[3]]
\end{verbatim}
\normalsize

Kết quả cho $Gr(2,5)$ cho thấy các lớp Schubert $S[\lambda]$ với $\lambda$ từ rỗng cho đến $(3,3)$ (phân hoạch lớn nhất $3,3$ ứng với lớp điểm). Các lớp sinh là $S[1], S[2], S[3]$ tương ứng với các điều kiện cơ bản nhất (các phân hoạch một ô trong hàng 1, 2, 3).

Tương tự, với một Grassmannian đẳng hướng:
\small
\begin{verbatim}
>>> ig = IsotropicGrassmannian(2,6)
>>> print(ig)
Type C; (k,n)=(1,3); IG(2,6); deg(q)=5

>>> ig.schub_classes()
[S[],
 S[1], S[1,1],
 S[2], S[2,1],
 S[3], S[3,1], S[3,2],
 S[4], S[4,1], S[4,2], S[4,3]]

>>> ig.point_class()
S[4,3]

>>> ig.generators()
[S[1], S[2], S[3], S[4]]
\end{verbatim}
\normalsize

Trong trường hợp $IG(2,6)$ (Grassmann đẳng hướng loại C), ta thấy xuất hiện các phân hoạch $k$-strict (ví dụ $4,3$ cho lớp điểm). Số lượng lớp nhiều hơn so với $Gr(2,5)$, phản ánh sự khác biệt về cấu trúc đồng điều của các không gian đẳng hướng.

\subsection{Phép nhân đồng điều và đồng điều lượng tử}
\texttt{SchubertPy} cung cấp các hàm để thực hiện những phép nhân trong vành đồng điều và đồng điều lượng tử:
\begin{itemize}
    \item \texttt{pieri(p, expr)} – tính tích $\sigma_{(p)} \cdot \texttt{expr}$ (với $p$ là một số nguyên dương, tạo lớp đặc biệt một hàng $p$ ô) bằng quy tắc Pieri.
    \item \texttt{giambelli(expr)} – biểu diễn lớp Schubert cho trước (dưới dạng biểu thức tổ hợp các $S[\lambda]$) về các lớp sinh sử dụng công thức Giambelli.
    \item \texttt{mult(expr1, expr2)} – tính tích của hai biểu thức \texttt{expr1} và \texttt{expr2} trong vành đồng điều.
    \item Các hàm tương tự cho vành đồng điều lượng tử: \texttt{qpieri}, \texttt{qgiambelli}, \texttt{qmult} – thực hiện phép nhân trong vành lượng tử (tức là có xét thêm phần $q$).
\end{itemize}

Tất cả các hàm trên nhận các biểu thức đa thức Schubert dưới dạng chuỗi (string) với ký hiệu $S[\lambda]$ cho lớp Schubert. Kết quả trả về cũng ở dạng biểu thức chuỗi. Dưới đây là một số ví dụ minh họa trên $Gr(2,5)$ (loại A) và $IG(2,6)$ (loại C):

Ví dụ trên $Gr(2,5)$:
\small
\begin{verbatim}
>>> gr = Grassmannian(2, 5)
>>> gr.pieri(1, 'S[2,1] - 7*S[3,2]')
S[2,2] + S[3,1] - 7*S[3,3]

>>> gr.giambelli('S[2,1] * S[2,1]')
S[1]^2*S[2]^2 - 2*S[1]*S[2]*S[3] + S[3]^2

>>> gr.mult('S[2,1]', 'S[2,1] + S[3,2]')
S[3,3]

>>> gr.qpieri(1, 'S[2,1] - 7*S[3,2]')
-7*S[1]*q + S[2,2] + S[3,1] - 7*S[3,3]

>>> gr.qgiambelli('S[2,1] * S[2,1]')
S[1]^2*S[2]^2 - 2*S[1]*S[2]*S[3] + S[3]^2

>>> gr.qmult('S[2,1]', 'S[2,1] + S[3,2]')
S[1]*q + S[2,1]*q + S[3]*q + S[3,3]
\end{verbatim}
\normalsize

Trong các kết quả trên:
\begin{itemize}
    \item \texttt{pieri(1, 'S[2,1] - 7*S[3,2]')} áp dụng quy tắc Pieri với $p=1$ (tức nhân với $\sigma_{(1)}$) cho biểu thức $S[2,1] - 7S[3,2]$.
    \item \texttt{giambelli('S[2,1] * S[2,1]')} biểu diễn lớp $S[2,1] * S[2,1]$ (tương ứng $\sigma_{(2,1)}$ bình phương) về các $S[1], S[2], S[3]$.
    \item \texttt{mult('S[2,1]', 'S[2,1] + S[3,2]')} tính tích của $S[2,1]$ với $S[2,1] + S[3,2]$ cho ra kết quả $S[3,3]$ (lớp điểm, nghĩa là hai lớp đó giao nhau tại hữu hạn 1 điểm).
    \item Các hàm \texttt{q*} cho kết quả trong đó có thể xuất hiện thêm các số hạng chứa $q$. Ví dụ $-7*S[1]*q$ trong kết quả của \texttt{qpieri} cho biết có một đường cong bậc 1 (hệ số $q$) xuất hiện trong phép nhân lượng tử.
\end{itemize}

Ví dụ tương tự trên $IG(2,6)$:
\small
\begin{verbatim}
>>> ig = IsotropicGrassmannian(2, 6)
>>> ig.pieri(1, 'S[2,1] - 7*S[3,2]')
2*S[3,1] + S[4] - 7*S[4,2]

>>> ig.giambelli('S[2,1]*S[2,1]')
S[1]^2*S[2]^2 - 2*S[1]*S[2]*S[3] + S[3]^2

>>> ig.mult('S[2,1]', 'S[2,1]+S[3,2]')
2*S[4,2]

>>> ig.qpieri(1, 'S[2,1] - 7*S[3,2]')
2*S[3,1] + S[4] - 7*S[4,2]

>>> ig.qgiambelli('S[2,1]*S[2,1]')
S[1]^2*S[2]^2 - 2*S[1]*S[2]*S[3] + S[3]^2

>>> ig.qmult('S[2,1]', 'S[2,1]+S[3,2]')
S[1]*q + S[3]*q + 2*S[4,2]
\end{verbatim}
\normalsize

Ta có thể thấy hầu hết các kết quả trên $IG(2,6)$ tương tự với trường hợp $Gr(2,5)$, ngoại trừ một số hệ số khác biệt (ví dụ: kết quả \texttt{mult} có hệ số 2, do tính chất đẳng hướng làm xuất hiện các bội số). Điều này minh họa sự phức tạp tăng thêm khi làm việc với các Grassmannian loại B/C/D. Giải thuật sử dụng trong \texttt{SchubertPy} được trình bày trong phần phụ lục \ref{appendix:pseudocode}.


\subsection{Chuyển đổi và tiện ích khác}
Ngoài các phép tính nhân, \texttt{SchubertPy} còn cung cấp một số hàm chuyển đổi hữu ích:
\begin{itemize}
    \item \texttt{dualize(expr)} – lấy mỗi lớp Schubert trong biểu thức \texttt{expr} thành lớp đối ngẫu Poincaré của nó (đối ngẫu trong vành đồng điều).
    \item \texttt{part2index(expr)} và \texttt{index2part(expr)} – chuyển đổi giữa ký hiệu phân hoạch (partition) và ký hiệu chỉ số (index set) biểu diễn cùng một lớp Schubert. Ký hiệu index set cho thấy điều kiện hình học của lớp Schubert (thông qua vị trí các điều kiện trong cờ).
\end{itemize}

Ví dụ:
\small
\begin{verbatim}
>>> gr = Grassmannian(2, 5)
>>> gr.dualize('S[1]+S[2]')
S[3,1] + S[3,2]

>>> gr.part2index('S[1]+S[2]')
S[2,5] + S[3,5]
\end{verbatim}
\normalsize

Kết quả:
\begin{itemize}
    \item \texttt{dualize('S[1]+S[2]')} trên $Gr(2,5)$ lấy đối ngẫu của lớp $S[1]$ và $S[2]$, cho ra $S[3,1] + S[3,2]$ (vì $(3,1)$ và $(3,2)$ là phân hoạch bổ sung của $(1)$ và $(2)$ trong lưới $3\times 2$).
    \item - \texttt{part2index('S[1]+S[2]')} chuyển $S[1]$ thành $S[2,5]$ và $S[2]$ thành $S[3,5]$. Ở đây ký hiệu $S[a,b]$ trong kết quả là ký hiệu index set: ví dụ $S[2,5]$ nghĩa là lớp Schubert gồm các mặt phẳng $P$ trong $\mathbb{C}^5$ sao cho $\dim(P \cap F_3) \ge 2$ và $\dim(P \cap F_4) \ge 3$ (điều kiện này tương ứng với phân hoạch $(1)$ ban đầu). Những ký hiệu này giúp người dùng kiểm tra điều kiện hình học tương ứng với mỗi phân hoạch nếu cần.
\end{itemize}


\subsection{Thuật toán Rim-Hook}

SchubertPy triển khai \textit{thuật toán remove rim-hook} (loại bỏ móc viền) để thực hiện phép nhân lượng tử giữa các lớp Schubert, đặc biệt là trong vành đồng điều lượng tử của không gian Grassmann. Thuật toán dựa trên công trình của Bertram, Ciocan-Fontanine và Fulton \cite{bertram1999quantum}, với các bước chính như sau:

\begin{itemize}
    \item Mở rộng phân hoạch $\lambda$ thành một phân hoạch mới trong hình chữ nhật $k \times (n-k)$ bằng cách cộng thêm các ô sao cho hợp lệ.    
    \item Loại bỏ các rim hooks. Một rim hook là một chuỗi các ô kề nhau, nằm dọc hoặc ngang ở biên, có độ dài chính xác $d$. Mỗi lần loại bỏ một rim hook, ta tiến một bước trong việc thu gọn phân hoạch.
    \item Mỗi lần loại bỏ một rim hook độ dài $d$, kết quả sẽ nhận được một hệ số $(-1)^{\mathrm{ht} - 1} q^d$, với $\mathrm{ht}$ là chiều cao của hook.
\end{itemize}


Tổng hợp lại tất cả các khả năng loại bỏ rim hook từ $\lambda$ để được các phân hoạch $\mu$, ta có công thức tổng quát cho tích lượng tử:

$$
\sigma_{\lambda} * \sigma_{\mu} = \sum_{\nu, d} c_{\lambda,\mu}^{\nu,d} \, q^d \, \sigma_{\nu}
$$

trong đó:
\begin{itemize}
    \item $\sigma_{\lambda}, \sigma_{\mu}, \sigma_{\nu}$ là các lớp Schubert,
    \item $c_{\lambda,\mu}^{\nu,d}$ là hệ số đếm số cách loại bỏ các rim hook có tổng độ dài $d$, dẫn đến phân hoạch $\nu$.
\end{itemize}

Ví dụ: trong đồng điều cổ điển, ta có:
\[
\sigma_{(2,1)} \cdot \sigma_{(2,1)} = \sigma_{(3,2)} + \sigma_{(4,1)},
\]
nhưng \(\sigma_{(3,2)}\) và \(\sigma_{(4,1)}\) đều quá lớn so với hình chữ nhật \(2 \times 2\), do đó trong đồng điều lượng tử, ta có:
\[
\sigma_{(2,1)} * \sigma_{(2,1)} = q \cdot \sigma_{(1,1)} + q \cdot \sigma_{(2)},
\]
Kết quả được biểu diễn bằng các phân hoạch nhỏ hơn kèm theo hiệu chỉnh lượng tử.

\section{Kết luận và hướng phát triển}
Chúng tôi đã trình bày tổng quan về phép tính Schubert – từ nền tảng lịch sử đến các khái niệm và công cụ tính toán cổ điển – và giới thiệu \texttt{SchubertPy} như một công cụ hỗ trợ mạnh mẽ cho việc tính toán trong lĩnh vực này. Phép tính Schubert là một lĩnh vực quan trọng của hình học đại số, với ứng dụng rộng rãi trong hình học liệt kê cổ điển, lý thuyết biểu diễn, và vật lý lý thuyết (thông qua đồng điều lượng tử và lý thuyết Gromov–Witten) \cite{Kontsevich_1994}. Việc có các công cụ phần mềm như \texttt{SchubertPy} giúp các nhà toán học và sinh viên dễ dàng thực hiện các phép tính thử nghiệm, kiểm chứng giả thuyết và khám phá các mẫu hình tổ hợp ẩn sau các con số đếm.

\texttt{SchubertPy} được thiết kế theo hướng mô-đun và mở rộng, tuân thủ các nguyên tắc lập trình hướng đối tượng, điều này cho phép dễ dàng bảo trì và phát triển thêm. Trong quá trình phát triển, chúng tôi đã chú trọng sử dụng \textit{type hints} của Python nhằm tăng độ rõ ràng của mã và tận dụng sự hỗ trợ của các trình kiểm tra tĩnh, cũng như xây dựng bộ kiểm thử tự động với \textit{Pytest} đạt độ bao phủ trên 80\% để đảm bảo độ tin cậy của các chức năng \cite{PythonTyping2023,PythonUnittestLib}. Gói \texttt{SchubertPy} hiện được phát hành dưới giấy phép mã nguồn mở GPL, cho phép cộng đồng tự do sử dụng và đóng góp.

Về định hướng tương lai, chúng tôi dự kiến một số hướng phát triển sau cho \texttt{SchubertPy}:
\begin{itemize}
    \item \textbf{Tích hợp với NumPy:} Tích hợp các cấu trúc dữ liệu và thuật toán của \texttt{SchubertPy} với thư viện \texttt{NumPy} để tăng hiệu quả tính toán, đặc biệt cho các phép tính ma trận lớn và nhiều phép nhân lặp lại \cite{githubschubertpy}.
    \item \textbf{Tối ưu hiệu năng:} Cải thiện hiệu năng của các thuật toán, ví dụ bằng cách cài đặt bộ nhớ đệm cho các kết quả đã tính để tránh tính toán lặp lại, hoặc chuyển các đoạn mã quan trọng sang C/C++ và liên kết với Python thông qua Cython/pybind11 \cite{githubschubertpy}.
    \item \textbf{Tích hợp với SageMath:} Kết nối \texttt{SchubertPy} vào hệ sinh thái SageMath \cite{sagemath}, cho phép người dùng SageMath gọi trực tiếp các hàm của \texttt{SchubertPy} như một module, từ đó kết hợp ưu điểm của cả hai nền tảng.
    \item \textbf{Tăng cường trực quan hóa:} Phát triển các chức năng trực quan hóa hình học, ví dụ vẽ hình ảnh của một vài trường hợp đa tạp Schubert hoặc mô phỏng vị trí các không gian con thỏa điều kiện, giúp người dùng có trực giác tốt hơn về các đối tượng đang nghiên cứu \cite{githubschubertpy}.
    \item \textbf{Mở rộng chức năng:} Bổ sung các thuật toán và hàm tính toán mới theo nhu cầu của người dùng và theo tiến triển của nghiên cứu về phép tính Schubert. Ví dụ, hiện chúng tôi đang xem xét triển khai thuật toán tính \textit{hệ số Littlewood-Richardson} trực tiếp, tính số \textit{Kostka} và các đại lượng quan trọng khác trong đại số tổ hợp liên quan \cite{githubschubertpy}. Mục tiêu lâu dài là làm cho \texttt{SchubertPy} trở thành một công cụ toàn diện cho cả phép tính Schubert cổ điển lẫn các mở rộng như đồng điều lượng tử và $K$-theory.
\end{itemize}

Những định hướng trên thể hiện cam kết của chúng tôi trong việc phát triển \texttt{SchubertPy} thành một công cụ mạnh mẽ, hiệu quả và thân thiện cho cộng đồng nghiên cứu phép tính Schubert. Chúng tôi hoan nghênh sự đóng góp ý kiến, báo lỗi và bổ sung tính năng từ cộng đồng mã nguồn mở nhằm cùng nhau hoàn thiện gói phần mềm này.

Về mặt toán học, phép tính Schubert vẫn tiếp tục là một lĩnh vực sôi động. Việc kết nối với các lý thuyết mới (chẳng hạn đối ngẫu gương, lý thuyết đồng điều lượng tử cho các đa tạp phức tổng quát, v.v.) mở ra nhiều câu hỏi và cơ hội nghiên cứu. Chúng tôi hy vọng rằng sự kết hợp giữa một nền tảng lý thuyết vững chắc và các công cụ tính toán tiện lợi sẽ giúp đẩy nhanh quá trình khám phá và đưa ra những kết quả mới trong tương lai.



\nocite{graysonschubert2}
\nocite{hiep2014schubert3}
\nocite{katz1992schubert}
\nocite{Kontsevich_1994}
\nocite{PythonTyping2023}
\nocite{PythonUnittestLib}
\nocite{SageMath2024}
\nocite{schubert1879kalkul}
\nocite{Hiep2014IntersectionTheory}
\nocite{Van2024BatBienDaiTap}

\vspace{7pt}
\bibliographystyle{alpha}
\bibliography{bibliography}


\appendix
\section*{Phụ lục: Pseudocode của Quy tắc Pieri cho các loại Grassmannian} \label{appendix:pseudocode}


\subsection*{Thuật toán Pieri Type A (Grassmannian thường): \texttt{pieriA\_inner} và các thuật toán bổ trợ}

\begin{algorithm}[H]
\caption{Pieri Rule Type A (\texttt{pieriA\_inner})}
\begin{algorithmic}[1]
\REQUIRE $i \in \mathbb{N}$, $\lambda = (\lambda_1, ..., \lambda_l)$, $k, n \in \mathbb{N}$
\ENSURE $\sum \sigma_\mu \in H^*(Gr(k,n))$
\STATE $\bar\lambda \gets \lambda$ nối thêm $0$ cho đủ độ dài $n-k$
\STATE $inner \gets \bar\lambda$
\STATE $outer \gets (k, \bar\lambda_1, ..., \bar\lambda_{n-k-1})$
\STATE $result \gets 0$
\STATE $\mu \gets$ $\mathtt{\_pieri\_fillA}$($inner$, $outer$, $0$, $i$)
\WHILE{$\mu \neq \emptyset$}
    \STATE $result \gets result + \sigma_{\mathtt{trim}(\mu)}$
    \STATE $\mu \gets$ $\mathtt{\_pieri\_itrA}$($\mu$, $inner$, $outer$)
\ENDWHILE
\RETURN $result$
\end{algorithmic}
\end{algorithm}

\begin{algorithm}[H]
\caption{Pieri Fill Type A (\texttt{\_pieri\_fillA})}
\begin{algorithmic}[1]
\REQUIRE $\lambda$, $inner$, $outer$ là các partition, $row\_index$, $p \in \mathbb{N}$
\ENSURE $\mu$ hoặc $\emptyset$
\IF{$\lambda = \emptyset$}
    \RETURN $\lambda$
\ENDIF
\STATE $res \gets \lambda.\mathtt{copy}()$
\STATE $pp \gets p$
\STATE $rr \gets row\_index$
\IF{$rr = 0$}
    \STATE $x \gets \min(outer[0], inner[0] + pp)$
    \STATE $res[0] \gets x$
    \STATE $pp \gets pp - x + inner[0]$
    \STATE $rr \gets 1$
\ENDIF
\WHILE{$rr < |\lambda|$}
    \STATE $x \gets \min(outer[rr], inner[rr] + pp, res[rr-1])$
    \STATE $res[rr] \gets x$
    \STATE $pp \gets pp - x + inner[rr]$
    \STATE $rr \gets rr + 1$
\ENDWHILE
\IF{$pp > 0$}
    \RETURN $\emptyset$
\ENDIF
\RETURN $res[:|\lambda|]$
\end{algorithmic}
\end{algorithm}

\begin{algorithm}[H]
\caption{Pieri Iterator Type A (\texttt{\_pieri\_itrA})}
\begin{algorithmic}[1]
\REQUIRE $\lambda$, $inner$, $outer$ là các partition
\ENSURE $\mu$ hoặc $\emptyset$
\IF{$\lambda = \emptyset$}
    \RETURN $\emptyset$
\ENDIF
\STATE $p \gets \lambda_{|\lambda|} - inner_{|\lambda|}$
\FOR{$r \gets |\lambda|-1$ down to $1$}
    \IF{$\lambda[r] > inner[r]$}
        \STATE $\mu \gets \lambda.\mathtt{copy}()$
        \STATE $\mu[r] \gets \mu[r] - 1$
        \STATE $\mu \gets$ $\mathtt{\_pieri\_fillA}$($\mu$, $inner$, $outer$, $r+1$, $p+1$)
        \IF{$\mu \neq \emptyset$}
            \RETURN $\mu$
        \ENDIF
    \ENDIF
    \STATE $p \gets p + \lambda[r] - inner[r]$
\ENDFOR
\RETURN $\emptyset$
\end{algorithmic}
\end{algorithm}

\subsection*{Thuật toán Pieri lượng tử Type A (Quantum): \texttt{qpieriA\_inner} và các thuật toán bổ trợ}

\begin{algorithm}[H]
\caption{Quantum Pieri Rule Type A (\texttt{qpieriA\_inner})}
\begin{algorithmic}[1]
\REQUIRE $i \in \mathbb{N}$, $\lambda = (\lambda_1, ..., \lambda_l)$, $k, n \in \mathbb{N}$
\ENSURE $\sum a_\mu \sigma_\mu + \sum b_\nu q^d \sigma_\nu \in QH^*(Gr(k,n))$
\STATE $result \gets$ $\mathtt{pieriA\_inner}$($i$, $\lambda$, $k$, $n$) \COMMENT{Số hạng cổ điển}
\IF{$|\lambda| = n-k$ và $\lambda_{n-k} > 0$}
    \IF{$k = 1$}
        \RETURN $q \cdot \sigma_{\emptyset}$
    \ENDIF
    \STATE $\lambda' \gets \{\lambda_j - 1 : \lambda_j > 1\}$
    \STATE $LC \gets$ $\mathtt{pieriA\_inner}$($i-1$, $\lambda'$, $k-1$, $n$)
    \STATE $f \gets$ ($\mu \mapsto$ $\mathtt{\_part\_star}$($\mu$, $k-1$))
    \STATE $Z \gets$ $\mathtt{apply\_lc}$($f$, $LC$)
    \STATE $result \gets result + q \cdot Z$
\ENDIF
\RETURN $result$
\end{algorithmic}
\end{algorithm}

\begin{algorithm}[H]
    \caption{Áp dụng biến đổi tuyến tính (\texttt{apply\_lc})}
    \begin{algorithmic}[1]
    \REQUIRE $f: \mathcal{P} \to \mathcal{H}$ là phép biến đổi partition, $L = \sum_\lambda a_\lambda \sigma_\lambda$ là tổ hợp tuyến tính các hàm Schur
    \ENSURE Tổ hợp tuyến tính mới $L' = \sum_\lambda a_\lambda \sigma_{f(\lambda)}$
    \STATE $result \gets 0$
    \IF{$L$ là tổ hợp tuyến tính}
        \FORALL{hạng tử $a_\lambda \sigma_\lambda$ trong $L$}
            \STATE $\mu \gets f(\lambda)$
            \STATE $result \gets result + a_\lambda \cdot \sigma_\mu$
        \ENDFOR
    \ELSE
        \STATE $result \gets f(L)$ \COMMENT{Áp dụng $f$ trực tiếp cho biểu thức đơn}
    \ENDIF
    \RETURN $result$
    \end{algorithmic}
    \end{algorithm}

\begin{algorithm}[H]
    \caption{Part Star Operation (\texttt{\_part\_star})}
    \begin{algorithmic}[1]
    \REQUIRE $\lambda = (\lambda_1, ..., \lambda_l)$, $cols \in \mathbb{N}$
    \ENSURE Hàm Schur hoặc $0$
    \STATE $result \gets \begin{cases}
        0 & \text{nếu } \lambda = \emptyset \text{ hoặc } \lambda_1 \neq cols \\
        \sigma_{\emptyset} & \text{nếu } |\lambda| = 1 \\
        \sigma_{(\lambda_2, ..., \lambda_l)} & \text{ngược lại}
    \end{cases}$
    \RETURN $result$
    \end{algorithmic}
    \end{algorithm}


% --- Pieri Type B Algorithms ---

\subsection*{Thuật toán Pieri Type B (Grassmannian trực giao lẻ) -- Python: \texttt{pieriB\_inner}}

\begin{algorithm}[H]
\caption{Pieri Rule Type B (\texttt{pieriB\_inner})}
\begin{algorithmic}[1]
\REQUIRE $p \in \mathbb{N}$, $\lambda = (\lambda_1, ..., \lambda_l)$, $k, n \in \mathbb{N}$
\ENSURE $\sum 2^{c(\lambda,\mu)-b} \sigma_\mu \in H^*(OG(k,2n+1))$
\STATE $result \gets 0$
\STATE $b \gets \begin{cases}0 & \text{nếu } p \leq k \\ 1 & \text{nếu } p > k\end{cases}$
\STATE $\mathcal{P} \gets$ PieriSet($p$, $\lambda$, $k$, $n$, $0$)
\FOR{mỗi $\mu \in \mathcal{P}$}
    \STATE $c \gets$ CountComps($\lambda$, $\mu$, false, $k$, $0$)
    \STATE $a_\mu \gets 2^{c-b}$
    \STATE $result \gets result + a_\mu \cdot \sigma_\mu$
\ENDFOR
\RETURN $result$
\end{algorithmic}
\end{algorithm}

\subsection*{Thuật toán Pieri lượng tử Type B (Quantum) -- Python: \texttt{qpieriB\_inner}}

\begin{algorithm}[H]
\caption{Quantum Pieri Rule Type B (\texttt{qpieriB\_inner})}
\begin{algorithmic}[1]
\REQUIRE $p \in \mathbb{N}$, $\lambda = (\lambda_1, ..., \lambda_l)$, $k, n \in \mathbb{N}$
\ENSURE $\sum a_\mu \sigma_\mu + \sum b_\nu q^d \sigma_\nu \in QH^*(OG(k,2n+1))$
\STATE $result \gets$ PieriB($p$, $\lambda$, $k$, $n$) \COMMENT{Số hạng cổ điển}
\IF{$k = 0$}
    \IF{$|\lambda| > 0$ và $\lambda_1 = n + k$}
        \STATE $T_1 \gets$ ApplyLC($\mu \mapsto$ PartStar($\mu$, $n+k$), PieriB($p$, $\lambda[2:]$, $k$, $n$))
        \STATE $result \gets result + q \cdot T_1$
    \ENDIF
\ELSE
    \IF{$|\lambda| = n-k$ và $\lambda_{n-k} > 0$}
        \STATE $T_2 \gets$ ApplyLC($\mu \mapsto$ PartTilde($\mu$, $n-k+1$, $n+k$), PieriB($p$, $\lambda$, $k$, $n+1$))
        \STATE $result \gets result + q \cdot T_2$
    \ENDIF
    \IF{$|\lambda| > 0$ và $\lambda_1 = n + k$}
        \STATE $T_3 \gets$ ApplyLC($\mu \mapsto$ PartStar($\mu$, $n+k$), PieriB($p$, $\lambda[2:]$, $k$, $n$))
        \STATE $result \gets result + q^2 \cdot T_3$
    \ENDIF
\ENDIF
\RETURN $result$
\end{algorithmic}
\end{algorithm}

% Các thuật toán hỗ trợ chung cho Pieri B đã được trình bày ở appendixE ("Thuật toán hỗ trợ chung cho Pieri B/C/D"). Vui lòng tham khảo appendixE cho các hàm: \texttt{pieri\_set}, \texttt{count\_comps}, \texttt{_pieri\_fill}, \texttt{_pieri\_itr}, \texttt{part\_conj}, \texttt{_part\_star}, \texttt{_part\_tilde}, \texttt{part\_clip}.
% --- Pieri Type C Algorithms ---

\subsection*{Thuật toán Pieri Type C (Grassmannian symplectic): \texttt{pieriC\_inner} và các thuật toán bổ trợ}

\begin{algorithm}[H]
\caption{Pieri Rule Type C (\texttt{pieriC\_inner})}
\begin{algorithmic}[1]
\REQUIRE $i \in \mathbb{N}$, $\lambda = (\lambda_1, ..., \lambda_l)$, $k, n \in \mathbb{N}$
\ENSURE $\sum 2^{c(\lambda,\mu)} \sigma_\mu \in H^*(IG(k,2n))$
\STATE $result \gets 0$
\STATE $\mathcal{P} \gets$ $\mathtt{pieri\_set}$($i$, $\lambda$, $k$, $n$, $0$)
\FOR{mỗi $\mu \in \mathcal{P}$}
    \STATE $c \gets$ $\mathtt{count\_comps}$($\lambda$, $\mu$, true, $k$, $0$)
    \STATE $a_\mu \gets 2^c$
    \STATE $result \gets result + a_\mu \cdot \sigma_\mu$
\ENDFOR
\RETURN $result$
\end{algorithmic}
\end{algorithm}

\subsection*{Thuật toán Pieri lượng tử Type C (Quantum): \texttt{qpieriC\_inner} và các thuật toán bổ trợ}

\begin{algorithm}[H]
\caption{Quantum Pieri Rule Type C (\texttt{qpieriC\_inner})}
\begin{algorithmic}[1]
\REQUIRE $i \in \mathbb{N}$, $\lambda = (\lambda_1, ..., \lambda_l)$, $k, n \in \mathbb{N}$
\ENSURE $\sum a_\mu \sigma_\mu + \sum b_\nu q^d \sigma_\nu \in QH^*(IG(k,2n))$
\STATE $result \gets$ $\mathtt{pieriC\_inner}$($i$, $\lambda$, $k$, $n$)
\IF{$|\lambda| = n-k$ và $\lambda_{n-k} > 0$}
    \STATE $\lambda' \gets \{\lambda_j - 1 : \lambda_j > 1\}$
    \STATE $LC \gets$ $\mathtt{pieriC\_inner}$($i-1$, $\lambda'$, $k-1$, $n$)
    \STATE $f \gets$ ($\mu \mapsto$ $\mathtt{\_part\_star}$($\mu$, $k-1$))
    \STATE $Z \gets$ $\mathtt{apply\_lc}$($f$, $LC$)
    \STATE $result \gets result + q \cdot Z$
\ENDIF
\RETURN $result$
\end{algorithmic}
\end{algorithm}


% --- Pieri Type D Algorithms ---

\subsection*{Thuật toán Pieri Type D (Grassmannian trực giao chẵn) -- Python: \texttt{pieriD\_inner}}

\begin{algorithm}[H]
\caption{Pieri Rule Type D (\texttt{pieriD\_inner})}
\begin{algorithmic}[1]
\REQUIRE $p \in \mathbb{Z}$, $\lambda = (\lambda_1, ..., \lambda_l)$, $k, n \in \mathbb{N}$
\ENSURE $\sum \text{Dcoef}(p,\lambda,\mu,t_\lambda,k,n) \sigma_\mu \in H^*(OG(k,2n+2))$
\STATE $result \gets 0$
\STATE $t_\lambda \gets$ TypeParameter($\lambda$, $k$)
\STATE $\mathcal{P} \gets$ PieriSet($|p|$, $\lambda$, $k$, $n$, $1$)
\FOR{mỗi $\mu \in \mathcal{P}$}
    \STATE $coef \gets$ Dcoef($p$, $\lambda$, $\mu$, $t_\lambda$, $k$, $n$)
    \STATE $result \gets result + coef$
\ENDFOR
\RETURN $result$
\end{algorithmic}
\end{algorithm}

\subsection*{Thuật toán hỗ trợ cho Pieri Type D}


% Thuật toán tính hệ số D (Type D Pieri coefficient) - Python: _dcoef

% _dcoef(p, lam, mu, tlam, k, n)
\begin{algorithm}[H]
\caption{Tính hệ số D: \texttt{\_dcoef}(p, $\lambda$, $\mu$, tlam, k, n)}
\begin{algorithmic}[1]
\REQUIRE $p \in \mathbb{Z}$, $\lambda, \mu$ là các partition, $tlam \in \{0,1,2\}$, $k, n \in \mathbb{N}$
\ENSURE Linear combination hệ số cho Pieri Type D
\STATE $\Delta \gets \begin{cases}
    0 & \text{nếu } |p| < k \\
    1 & \text{các trường hợp khác}
\end{cases}$
\STATE $cc \gets \text{count\_comps}(\lambda, \mu, \text{False}, k, 1) - \Delta$
\IF{$cc \geq 0$}
    \STATE $result \gets \begin{cases}
        \sigma_\mu & \text{nếu } k \notin \mu \text{ hoặc } tlam = 1 \\
        \sigma_{\mu + [0]} & \text{nếu } tlam = 2 \\
        \sigma_\mu + \sigma_{\mu + [0]} & \text{ngược lại}
    \end{cases}$
    \RETURN $2^{cc} \cdot result$
\ENDIF
% Tie breaking
\STATE $\Delta \gets \begin{cases}
    1 & \text{nếu } p < 0 \\
    0 & \text{các trường hợp khác}
\end{cases}$
\STATE $h \gets k + tlam + \Delta$
\IF{$tlam = 0$ và $k \in \mu$}
    \RETURN $\begin{cases}
        \sigma_\mu & \text{nếu } h = 0 \\
        \sigma_{\mu + [0]} & \text{nếu } h \neq 0
    \end{cases}$
\ENDIF
\RETURN $\begin{cases}
    0 & \text{nếu } h = 0 \\
    \sigma_{\mu + [0]} & \text{nếu } tlam = 2 \text{ và } k \in \mu \\
    \sigma_\mu & \text{ngược lại}
\end{cases}$
\end{algorithmic}
\end{algorithm}

% Thuật toán tie-breaking value (Python: _tie_breaking_value)
\begin{algorithm}[H]
\caption{Tính tie-breaking value: $\mathtt{\_tie\_breaking\_value}(\lambda, \mu, k, tlam, p)$}
\begin{algorithmic}[1]
\REQUIRE $\lambda, \mu$ là partition, $k, tlam, p \in \mathbb{N}$
\ENSURE $h \in \{0,1\}$
\STATE $h \gets k + tlam + \begin{cases}
    1 & \text{nếu } p < 0 \\
    0 & \text{ngược lại}
\end{cases}$
\STATE $pmu \gets 0$
\FOR{$i$ từ $|\mu|-1$ đến $0$}
    \STATE $lami \gets \begin{cases}
        \lambda[i] & \text{nếu } i < |\lambda| \\
        0 & \text{ngược lại}
    \end{cases}$
    \IF{$lami < \min(\mu[i], k)$}
        \STATE $h \gets h - (\min(\mu[i], k) - \max(pmu, lami))$
    \ENDIF
    \STATE $pmu \gets \mu[i]$
\ENDFOR
\RETURN $h \bmod 2$
\end{algorithmic}
\end{algorithm}

\begin{algorithm}[H]
\caption{Pieri Set Generation (\texttt{pieri\_set})}
\begin{algorithmic}[1]
\REQUIRE $p \in \mathbb{Z}$, $\lambda = (\lambda_1, ..., \lambda_l)$, $k, n \in \mathbb{N}$
\ENSURE Tập hợp các partition tương ứng với quy tắc Pieri Type D
\STATE $\mathcal{P} \gets \emptyset$
\STATE $\mu \gets$ Partition chuẩn với độ dài $k$
\WHILE{$\mu \neq \emptyset$}
    \STATE $\mathcal{P} \gets \mathcal{P} \cup \{\mu\}$
    \STATE $\mu \gets$ PieriIterD($\mu$, $\lambda$, $k$, $n$)
\ENDWHILE
\RETURN $\mathcal{P}$
\end{algorithmic}
\end{algorithm}

\begin{algorithm}[H]
\caption{Count Connected Components (\texttt{count\_comps}, skipfirst=false, d=1)}
\begin{algorithmic}[1]
\REQUIRE $\lambda, \mu$ là các partition, $flag \in \{true, false\}$, $k, d \in \mathbb{N}$
\ENSURE Số lượng thành phần liên thông
\STATE $c \gets 0$
\STATE $used \gets$ đánh dấu tất cả các ô của $\lambda$ là chưa sử dụng
\FOR{mỗi ô $u$ trong $\lambda$}
    \IF{$used[u] = false$}
        \STATE $c \gets c + 1$
        \STATE Đánh dấu tất cả các ô liên thông với $u$ là đã sử dụng
    \ENDIF
\ENDFOR
\RETURN $c$
\end{algorithmic}
\end{algorithm}

\begin{algorithm}[H]
\caption{Pieri Fill Type B/C/D (\texttt{\_pieri\_fill})}
\begin{algorithmic}[1]
\REQUIRE $\lambda$, $inner$, $outer$ là các partition, $row\_index$, $p \in \mathbb{N}$
\ENSURE $\mu$ hoặc $\emptyset$
\IF{$\lambda = \emptyset$}
    \RETURN $\lambda$
\ENDIF
\STATE $res \gets \lambda.copy()$
\STATE $pp \gets p$
\STATE $rr \gets row\_index$
\IF{$ rr = 0 $}
    \STATE $x \gets \min(outer[0], inner[0] + pp)$
    \STATE $res[0] \gets x$
    \STATE $pp \gets pp - x + inner[0]$
    \STATE $rr \gets 1$
\ENDIF
\WHILE{$rr < |\lambda|$}
    \STATE $x \gets \min(outer[rr], inner[rr] + pp, res[rr-1])$
    \STATE $res[rr] \gets x$
    \STATE $pp \gets pp - x + inner[rr]$
    \STATE $rr \gets rr + 1$
\ENDWHILE
\IF{$pp > 0$}
    \RETURN $\emptyset$
\ENDIF
\RETURN $res[:|\lambda|]$
\end{algorithmic}
\end{algorithm}

\begin{algorithm}[H]
\caption{Pieri Iterator Type B/C/D (\texttt{\_pieri\_itr})}
\begin{algorithmic}[1]
\REQUIRE $\lambda$, $inner$, $outer$ là các partition
\ENSURE $\mu$ hoặc $\emptyset$
\IF{$\lambda = \emptyset$}
    \RETURN $\emptyset$
\ENDIF
\STATE $p \gets \lambda_{|\lambda|} - inner_{|\lambda|}$
\FOR{$r \gets |\lambda|-1$ down to $1$}
    \IF{$\lambda[r] > inner[r]$}
        \STATE $\mu \gets \lambda.copy()$
        \STATE $\mu[r] \gets \mu[r] - 1$
        \STATE $\mu \gets$ PieriFillB($\mu$, $inner$, $outer$, $r+1$, $p+1$)
        \IF{$\mu \neq \emptyset$}
            \RETURN $\mu$
        \ENDIF
    \ENDIF
    \STATE $p \gets p + \lambda[r] - inner[r]$
\ENDFOR
\RETURN $\emptyset$
\end{algorithmic}
\end{algorithm}

\begin{algorithm}[H]
\caption{Partition Conjugate (\texttt{part\_conj})}
\begin{algorithmic}[1]
\REQUIRE $\lambda = (\lambda_1, ..., \lambda_l)$
\ENSURE $\lambda'$ là phân hoạch liên hợp
\STATE $\lambda' \gets (\lambda_l, ..., \lambda_1)$
\RETURN $\lambda'$
\end{algorithmic}
\end{algorithm}


\begin{algorithm}[H]
\caption{Dualization Operation (\texttt{dualize})}
\begin{algorithmic}[1]
\REQUIRE $\lambda = (\lambda_1, ..., \lambda_l)$
\ENSURE $\lambda^*$ là phân hoạch đối ngẫu
\STATE $\lambda^* \gets (\lambda_l, ..., \lambda_1)$
\RETURN $\lambda^*$
\end{algorithmic}
\end{algorithm}

\begin{algorithm}[H]
\caption{Index Dualization (\texttt{dualize\_index\_inner})}
\begin{algorithmic}[1]
\REQUIRE $\lambda = (\lambda_1, ..., \lambda_l)$, $i \in \mathbb{N}$
\ENSURE $\lambda'$ với chỉ số đã được đối ngẫu
\STATE $\lambda' \gets \lambda$
\STATE $\lambda'[i] \gets \lambda[i] - 1$
\RETURN $\lambda'$
\end{algorithmic}
\end{algorithm}

\begin{algorithm}[H]
\caption{Type Swap Operation (\texttt{type\_swap})}
\begin{algorithmic}[1]
\REQUIRE $\lambda = (\lambda_1, ..., \lambda_l)$
\ENSURE $\lambda'$ với kiểu đã được hoán đổi
\STATE $\lambda' \gets \emptyset$
\FOR{$i = 1$ đến $|\lambda|$}
    \IF{$i$ lẻ}
        \STATE $\lambda'[i] \gets \lambda[i] + 1$
    \ELSE
        \STATE $\lambda'[i] \gets \lambda[i] - 1$
    \ENDIF
\ENDFOR
\RETURN $\lambda'$
\end{algorithmic}
\end{algorithm}

\begin{algorithm}[H]
\caption{Partition Type Swap (\texttt{type\_swap\_inner})}
\begin{algorithmic}[1]
\REQUIRE $\lambda = (\lambda_1, ..., \lambda_l)$
\ENSURE $\lambda'$ với kiểu đã được hoán đổi
\STATE $\lambda' \gets \emptyset$
\FOR{$i = 1$ đến $|\lambda|$}
    \IF{$i$ lẻ}
        \STATE $\lambda'[i] \gets \lambda[i] - 1$
    \ELSE
        \STATE $\lambda'[i] \gets \lambda[i] + 1$
    \ENDIF
\ENDFOR
\RETURN $\lambda'$
\end{algorithmic}
\end{algorithm}

\begin{algorithm}[H]
\caption{Quantum Helper: ToSchurFromIntnMu (\texttt{\_toSchurFromIntnMu})}
\begin{algorithmic}[1]
\REQUIRE $i \in \mathbb{N}$, $\lambda = (\lambda_1, ..., \lambda_l)$, $k, n \in \mathbb{N}$
\ENSURE $\sum a_\mu \sigma_\mu + \sum b_\nu q^d \sigma_\nu \in QH^*(Gr(k,n))$
\STATE $result \gets$ PieriA($i$, $\lambda$, $k$, $n$)
\IF{$|\lambda| = n-k$ và $\lambda_{n-k} > 0$}
    \IF{$k = 1$}
        \RETURN $q \cdot \sigma_{\emptyset}$
    \ENDIF
    \STATE $\lambda' \gets \{\lambda_j - 1 : \lambda_j > 1\}$
    \STATE $LC \gets$ PieriA($i-1$, $\lambda'$, $k-1$, $n$)
    \STATE $Z \gets$ ApplyLC($\mu \mapsto$ PartStar($\mu$, $k-1$), $LC$)
    \STATE $result \gets result + q \cdot Z$
\ENDIF
\RETURN $result$
\end{algorithmic}
\end{algorithm}

\subsection*{Thuật toán Pieri lượng tử Type D (Quantum) -- Python: \texttt{qpieriD\_inner}}

\begin{algorithm}[H]
\caption{Quantum Pieri Rule Type D (\texttt{qpieriD\_inner})}
\begin{algorithmic}[1]
\REQUIRE $p \in \mathbb{Z}$, $\lambda = (\lambda_1, ..., \lambda_l)$, $k, n \in \mathbb{N}$
\ENSURE $\sum a_\mu \sigma_\mu + \sum b_\nu q^d \sigma_\nu \in QH^*(OG(k,2n+2))$
\STATE $result \gets$ PieriD($p$, $\lambda$, $k$, $n$) \COMMENT{Số hạng cổ điển}
% TODO: Bổ sung chi tiết các điều kiện lượng tử đặc thù cho Type D nếu có
\RETURN $result$ % (hoặc thêm các số hạng lượng tử nếu có)
\end{algorithmic}
\end{algorithm}
% % --- Thuật toán hỗ trợ chung cho Pieri B/C/D ---

% \subsection*{Thuật toán hỗ trợ chung cho Pieri B/C/D}

% \begin{algorithm}[H]
% \caption{Partition Conjugate (\texttt{part\_conj})}
% \begin{algorithmic}[1]
% \REQUIRE $\lambda = (\lambda_1, ..., \lambda_l)$
% \ENSURE $\lambda'$ là phân hoạch liên hợp
% \STATE $\lambda' \gets (\lambda_l, ..., \lambda_1)$
% \RETURN $\lambda'$
% \end{algorithmic}
% \end{algorithm}








% \begin{algorithm}[H]
%     \caption{Partition Conjugate (\texttt{part\_conj})}
%     \begin{algorithmic}[1]
%     \REQUIRE $\lambda = (\lambda_1, ..., \lambda_l)$
%     \ENSURE $\lambda'$ là phân hoạch liên hợp
%     \STATE $\lambda' \gets (\lambda_l, ..., \lambda_1)$
%     \RETURN $\lambda'$
%     \end{algorithmic}
%     \end{algorithm}
    
    
    
    
%     \begin{algorithm}[H]
%     \caption{Index Dualization (\texttt{dualize\_index\_inner})}
%     \begin{algorithmic}[1]
%     \REQUIRE $\lambda = (\lambda_1, ..., \lambda_l)$, $i \in \mathbb{N}$
%     \ENSURE $\lambda'$ với chỉ số đã được đối ngẫu
%     \STATE $\lambda' \gets \lambda$
%     \STATE $\lambda'[i] \gets \lambda[i] - 1$
%     \RETURN $\lambda'$
%     \end{algorithmic}
%     \end{algorithm}
    
    
    
%     \begin{algorithm}[H]
%     \caption{Partition Type Swap (\texttt{type\_swap\_inner})}
%     \begin{algorithmic}[1]
%     \REQUIRE $\lambda = (\lambda_1, ..., \lambda_l)$
%     \ENSURE $\lambda'$ với kiểu đã được hoán đổi
%     \STATE $\lambda' \gets \emptyset$
%     \FOR{$i = 1$ đến $|\lambda|$}
%         \IF{$i$ lẻ}
%             \STATE $\lambda'[i] \gets \lambda[i] - 1$
%         \ELSE
%             \STATE $\lambda'[i] \gets \lambda[i] + 1$
%         \ENDIF
%     \ENDFOR
%     \RETURN $\lambda'$
%     \end{algorithmic}
%     \end{algorithm}
    
    
% --- Giambelli Algorithms ---

\subsection*{Thuật toán Giambelli (mọi loại Grassmannian) -- Python: \texttt{giambelli}, \texttt{giambelli\_rec}, \texttt{giambelli\_rec\_inner}}

\begin{algorithm}[H]
\caption{Giambelli Rule (Classical) (\texttt{giambelli})}
\begin{algorithmic}[1]
\REQUIRE $lc$ là LinearCombination các lớp Schubert
\ENSURE Đa thức các lớp Schubert đặc biệt
\STATE $lc \gets$ LinearCombination($lc$)
\STATE $pieri \gets \lambda i, p \to \_pieri(i, p, \_k, \_n)$
\RETURN GiambelliRec($lc$, $pieri$, $\_k$)
\end{algorithmic}
\end{algorithm}

\begin{algorithm}[H]
\caption{Giambelli Recursive Driver (\texttt{giambelli\_rec})}
\begin{algorithmic}[1]
\REQUIRE $lc$ là LinearCombination, $pieri$ là function, $k \in \mathbb{N}$
\ENSURE Đa thức các lớp Schubert đặc biệt
\STATE $lc \gets$ LinearCombination($lc$)
\RETURN ApplyLC($\lambda \mapsto$ GiambelliRecInner($\lambda$, $pieri$, $k$), $lc$)
\end{algorithmic}
\end{algorithm}

\begin{algorithm}[H]
\caption{Giambelli Recursive Inner Core (\texttt{giambelli\_rec\_inner})}
\begin{algorithmic}[1]
\REQUIRE $\lambda = (\lambda_1, ..., \lambda_l)$, $pieri$ là function, $k \in \mathbb{N}$
\ENSURE Đa thức các lớp Schubert đặc biệt
\IF{$|\lambda| = 0$}
    \RETURN $1$
\ENDIF
\STATE $p \gets \lambda_1$
\IF{$p = k$ và $\lambda_l = 0$}
    \STATE $p \gets -k$
\ENDIF
\STATE $\lambda' \gets \lambda[2:]$
\IF{$\lambda_l = 0$ và $\lambda_2 < k$}
    \STATE $\lambda' \gets \lambda[2:-1]$
\ENDIF
\STATE $pieriExpansion \gets$ pieri($p$, $\lambda'$)
\STATE $stuff \gets$ pieriExpansion $- \sigma_\lambda$
\STATE $a \gets$ GiambelliRecInner($\lambda'$, $pieri$, $k$)
\STATE $b \gets$ GiambelliRec($stuff$, $pieri$, $k$)
\STATE $result \gets$ Num2Spec($p$) $\cdot a - b$
\RETURN Expand($result$)
\end{algorithmic}
\end{algorithm}

\subsection*{Helper Functions cho Giambelli}

\begin{algorithm}[H]
\caption{Special Schubert to Number (\texttt{spec2num})}
\begin{algorithmic}[1]
\REQUIRE $sc$ là special Schubert class
\ENSURE $p \in \mathbb{Z}$
\IF{$sc$ không phải Schubert class}
    \STATE ERROR("special schubert class expected")
\ENDIF
\IF{$|sc| > 1$ và ($\_type \neq D$ hoặc $sc[2] \neq 0$)}
    \STATE ERROR("single part expected")
\ENDIF
\IF{$|sc| > 1$}
    \RETURN $-sc[1]$
\ELSE
    \RETURN $sc[1]$
\ENDIF
\end{algorithmic}
\end{algorithm}

\begin{algorithm}[H]
\caption{Number to Special Schubert (\texttt{num2spec})}
\begin{algorithmic}[1]
\REQUIRE $p \in \mathbb{Z}$
\ENSURE Special Schubert class
\IF{$p > 0$}
    \RETURN $\sigma_p$
\ELSE
    \RETURN $\sigma_{-p,0}$
\ENDIF
\end{algorithmic}
\end{algorithm}

\begin{algorithm}[H]
\caption{Apply Function to Linear Combination (\texttt{apply\_lc})}
\begin{algorithmic}[1]
\REQUIRE $f$ là function, $lc$ là LinearCombination
\ENSURE LinearCombination mới
\STATE \textit{(Xem chi tiết ở phần helper Type A)}
\end{algorithmic}
\end{algorithm}

\subsection*{Thuật toán Giambelli lượng tử (Quantum) -- Python: \texttt{qgiambelli}}

\begin{algorithm}[H]
\caption{Quantum Giambelli Rule (\texttt{qgiambelli})}
\begin{algorithmic}[1]
\REQUIRE $lc$ là LinearCombination các lớp Schubert
\ENSURE Đa thức trong QH* với tham số lượng tử $q$
\STATE $lc \gets$ LinearCombination($lc$)
\STATE $qpieri \gets \lambda i, p \to \_qpieri(i, p, \_k, \_n)$
\RETURN GiambelliRec($lc$, $qpieri$, $\_k$)
\end{algorithmic}
\end{algorithm}

% --- Littlewood-Richardson Algorithms ---

\subsection*{Thuật toán Littlewood-Richardson (mọi loại Grassmannian) -- Python: \texttt{mult}, \texttt{act}, \texttt{act\_lc}}

\begin{algorithm}[H]
\caption{Littlewood-Richardson Multiplication (Classical) (\texttt{mult})}
\begin{algorithmic}[1]
\REQUIRE $lc_1, lc_2$ là LinearCombination các lớp Schubert
\ENSURE LinearCombination kết quả tích
\STATE $lc_1 \gets$ LinearCombination($lc_1$)
\STATE $lc_2 \gets$ LinearCombination($lc_2$)
\STATE $polynomial \gets$ Giambelli($lc_1$)
\STATE $result \gets$ Act($polynomial$, $lc_2$)
\RETURN $result$
\end{algorithmic}
\end{algorithm}

\begin{algorithm}[H]
\caption{Action of Polynomial via Pieri Rules (\texttt{act})}
\begin{algorithmic}[1]
\REQUIRE $expr$ là đa thức các lớp đặc biệt, $lc$ là LinearCombination
\ENSURE Kết quả tác động $expr$ lên $lc$
\STATE $expr \gets$ LinearCombination($expr$).expr
\STATE $lc \gets$ LinearCombination($lc$)
\STATE $pieri\_func \gets \lambda(i,p) \to \_pieri(i, p, \_k, \_n)$
\RETURN ActLC($expr$, $lc$, $pieri\_func$)
\end{algorithmic}
\end{algorithm}

\begin{algorithm}[H]
\caption{Action Linear Combination Core (\texttt{act\_lc})}
\begin{algorithmic}[1]
\REQUIRE $expr$ là đa thức, $lc$ là LinearCombination, $pieri$ là hàm Pieri
\ENSURE Kết quả tác động
\STATE $vars \gets$ Các biến trong $expr$ (bỏ $q$)
\IF{$|vars| = 0$}
    \RETURN $expr \cdot lc$
\ENDIF
\STATE $v \gets$ biến đầu tiên trong $vars$
\STATE $i \gets$ Spec2Num($v$)
\STATE $expr_0 \gets$ $expr$ với $v=0$
\STATE $expr_1 \gets$ $(expr - expr_0)/v$
\STATE $term1 \gets$ ApplyLC($p \to$ pieri($i$, $p$), ActLC($expr_1$, $lc$, $pieri$))
\STATE $term2 \gets$ ActLC($expr_0$, $lc$, $pieri$)
\RETURN $term1 + term2$
\end{algorithmic}
\end{algorithm}

\subsection*{Thuật toán Littlewood-Richardson lượng tử (Quantum) -- Python: \texttt{qmult}, \texttt{qact}}

\begin{algorithm}[H]
\caption{Quantum Littlewood-Richardson Multiplication (\texttt{qmult})}
\begin{algorithmic}[1]
\REQUIRE $lc_1, lc_2$ là LinearCombination các lớp Schubert
\ENSURE Quantum product $lc_1 \cdot lc_2$ (có thể có $q$)
\STATE $lc_1 \gets$ LinearCombination($lc_1$)
\STATE $lc_2 \gets$ LinearCombination($lc_2$)
\STATE $q\_polynomial \gets$ QGiambelli($lc_1$)
\STATE $result \gets$ QAct($q\_polynomial$, $lc_2$)
\RETURN $result$
\end{algorithmic}
\end{algorithm}
\section*{Thuật toán Remove Rim Hooks (\texttt{remove\_rim\_hooks})}

\begin{algorithm}[H]
\caption{Remove Rim Hooks (\texttt{remove\_rim\_hooks})}
\begin{algorithmic}[1]
\REQUIRE $\lambda = (\lambda_1, ..., \lambda_l)$ là partition, $rim\_size \in \mathbb{N}$, $acceptable\_grid = (nrow, ncol)$
\ENSURE $(\lambda', \text{num\_rim\_hooks}, \text{total\_height})$
\IF{$\lambda$ rỗng hoặc $rim\_size \leq 0$}
    \RETURN $(\lambda, 0, 0)$
\ENDIF
\STATE $current\_partition \gets \lambda$
\STATE $total\_rim\_hooks\_removed \gets 0$
\WHILE{True}
    \STATE Tìm rim hook kích thước $rim\_size$ có thể loại khỏi $current\_partition$
    \IF{tìm được}
        \STATE Loại rim hook đó khỏi $current\_partition$
        \STATE $total\_rim\_hooks\_removed \gets total\_rim\_hooks\_removed + 1$
        \IF{partition mới phù hợp với $acceptable\_grid$}
            \RETURN (partition mới, $total\_rim\_hooks\_removed$, tổng chiều cao)
        \ENDIF
        \IF{không còn thay đổi}
            \RETURN $(\emptyset, total\_rim\_hooks\_removed, tổng chiều cao)$
        \ENDIF
        \STATE Cập nhật $current\_partition$
    \ELSE
        \RETURN $(0, 0, 0)$
    \ENDIF
\ENDWHILE
\end{algorithmic}
\end{algorithm}

\section*{Thuật toán Quantum Schubert với Rim Hook Removal}

\begin{algorithm}[H]
\caption{Quantum Schubert Class Calculation via Rim Hook Removal}
\begin{algorithmic}[1]
\REQUIRE $\lambda$ là partition, $n, k$ là tham số Grassmannian
\ENSURE Biểu thức Schubert lượng tử $QH^*(Gr(k,n))$
\STATE $acceptable\_grid \gets (n-k, k)$
\STATE $rim\_size \gets n$
\IF{$\lambda$ nằm trong $acceptable\_grid$}
    \RETURN $\sigma_\lambda$
\ENDIF
\STATE $(\lambda', q\_num, height) \gets$ remove\_rim\_hooks($\lambda$, $rim\_size$, $acceptable\_grid$)
\STATE $sign \gets (-1)^{height-kp}$
\IF{$q\_num > 0$}
    \RETURN $\sigma_{\lambda'} \cdot q^{q\_num} \cdot sign$
\ELSIF{$\lambda' = \emptyset$}
    \RETURN $0$
\ELSE
    \RETURN $\sigma_{\lambda'} \cdot sign$
\ENDIF
\end{algorithmic}
\end{algorithm}

% Quantum Schubert class calculation using remove_rim_hooks
\begin{algorithm}[H]
\caption{quantum\_schubert\_class (Python: grassmannian.quantum\_schubert\_class)}
\begin{algorithmic}[1]
\REQUIRE Partition $\lambda$, quantum grid size $n$, quantum parameter $q$
\ENSURE Quantum Schubert class $\sigma_\lambda$ in $QH^*(Gr(k,n))$
\STATE $\text{result} \gets 0$
\FOR{each $\mu$ in all partitions fitting in $k \times (n-k)$ grid}
    \STATE $(\nu, q\_power, \text{sign}) \gets \text{remove\_rim\_hooks}(\lambda, n)$
    \IF{$\nu$ fits in $k \times (n-k)$}
        \STATE $\text{result} \gets \text{result} + \text{sign} \cdot q^{q\_power} \cdot \sigma_\nu$
    \ENDIF
\ENDFOR
\RETURN $\text{result}$
\end{algorithmic}
\end{algorithm}


][/]
\end{document}


